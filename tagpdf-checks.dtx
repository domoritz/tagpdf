% \iffalse meta-comment
%
%% File: tagpdf-checks.dtx
%
% Copyright (C) 2019-2023 Ulrike Fischer
%
% It may be distributed and/or modified under the conditions of the
% LaTeX Project Public License (LPPL), either version 1.3c of this
% license or (at your option) any later version.  The latest version
% of this license is in the file
%
%    https://www.latex-project.org/lppl.txt
%
% This file is part of the "tagpdf bundle" (The Work in LPPL)
% and all files in that bundle must be distributed together.
%
% -----------------------------------------------------------------------
%
% The development version of the bundle can be found at
%
%    https://github.com/u-fischer/tagpdf
%
% for those people who are interested.
%
%
%<*driver>
\DocumentMetadata{}
\documentclass{l3doc}
\usepackage{array,booktabs,caption}
\hypersetup{pdfauthor=Ulrike Fischer,
 pdftitle=tagpdf-checks module (tagpdf)}
\begin{document}
  \DocInput{\jobname.dtx}
\end{document}
%</driver>
% \fi
% \title{^^A
%   The \pkg{tagpdf-checks} module\\ Messages and check code   ^^A
%   \\ Part of the tagpdf package
% }
%
% \author{^^A
%  Ulrike Fischer\thanks
%    {^^A
%      E-mail:
%        \href{mailto:fischer@troubleshooting-tex.de}
%          {fischer@troubleshooting-tex.de}^^A
%    }^^A
% }
%
% \date{Version 0.98f, released 2023-04-24}
% \maketitle
% \begin{documentation}
% \section{Commands}
% \begin{function}[pTF,EXP]{\tag_if_active:}
% This command tests if tagging is active. It only gives true if all tagging has
% been activated, \emph{and} if tagging hasn't been stopped locally.
% \end{function}
% \begin{function}[EXP]{\tag_get:n}
%  \begin{syntax}
%  \cs{tag_get:n}\Arg{keyword}
%  \end{syntax}
% This is a generic command to retrieve data for the current structure or
% mc-chunk. Currently
% the only sensible values for the argument \meta{keyword}
% are |mc_tag|, |struct_tag|, |struct_id| and |struct_num|.
% \end{function}
%
% \section{Description of log messages}
% \subsection{\cs{ShowTagging} command}
%  \begin{tabular}{lll}
%   Argument                          & type           & note\\
%  \cs{ShowTagging}{mc-data = num}    & log+term       & lua-only\\
%  \cs{ShowTagging}{mc-current} & log+term       & \\
%  \cs{ShowTagging}{struck-stack= [log\verb+|+show]}  & log or term+stop \\
%  \end{tabular}
%
% \subsection{Messages in checks and commands}
% \scriptsize
% \begin{tabular}{llll}
%   command                & message  & action   & remark\\
%   |\@@_check_structure_has_tag:n|
% & |struct-missing-tag|
% & error
% \\
%   |\@@_check_structure_tag:N|
% & |role-unknown-tag|
% & warning
% \\
%   |\@@_check_info_closing_struct:n|
% & |struct-show-closing|
% & info
% & log-level>0
% \\
%   |\@@_check_no_open_struct:|
% & struct-faulty-nesting
% & error
% & TODO: error only with 1?
% \\
%   |\@@_check_struct_used:n|
% & |struct-used-twice|
% & warning
% \\
%   |\@@_check_add_tag_role:nn|
% & |role-missing|, |role-tag|, |role-unknown|
% & warning, info (>0), warning
% \\
%   |\@@_check_mc_if_nested:|,
% & |mc-nested|
% & warning
% \\
%   |\@@_check_mc_if_open:|
% & |mc-not-open|
% & warning
% & only generic (?)
% \\
%   |\@@_check_mc_pushed_popped:nn|
% & |mc-pushed|, |mc-popped|
% & info (2), info+|seq_log| (>2)
% \\
%   |\@@_check_mc_tag:N|
% & |mc-tag-missing|, |role-unknown-tag|
% & error (missing), warning (unknown).
% \\
%   |\@@_check_mc_used:n|
% & |mc-used-twice|
% & warning
% & TODO: review the sense of this test!
% \\
%   |\@@_check_show_MCID_by_page:|
% &
% &
% & currently unused
% \\
%   |\tag_mc_use:n|
% & |mc-label-unknown|, |mc-used-twice|
% & warning
% & in mc-shared
% \\
%   |\role_add_tag:nn|
% & |new-tag|
% &  info (>0)
% & in roles
% \\
% & |sys-no-interwordspace|
% &  warning
% & space module, only xetex/dvi
% \\
%   |\@@_struct_write_obj:n|
% & |struct-no-objnum|
% & error
% & in struct module
% \\
%   |\tag_struct_begin:n|
% & |struct-faulty-nesting|
% & error
% \\
%   |\@@_struct_insert_annot:nn|
% & |struct-faulty-nesting|
% & error
% \\
%   |tag_struct_use:n|
% & |struct-label-unknown|
% &  warning
% \\
%   |attribute-class|, |attribute|
% & |attr-unknown|
% & error
% \\
%  |\@@_tree_fill_parenttree:|
% & |tree-mcid-index-wrong|
% & warning
% TODO: should trigger a standard  rerun message.
% \\
%  in |enddocument/info|-hook
% & |para-hook-count-wrong|
% & error (warning?)
% \end{tabular}
%
% \normalsize
% \subsection{Messages from the ptagging code}
% A few messages are issued in generic
% mode from the code which reinserts missing TMB/TME.
% This is currently done if log-level is larger than zero.
% TODO: reconsider log-level and messages when this code
% settles down.
%
%
% \subsection{Warning messages from the lua-code}
% The messages are triggered if the log-level is at least equal to the number.
%
% \begin{tabular}[t]{>{\ttfamily}lll}
% message & log-level & remark \\\hline
% WARN TAG-NOT-TAGGED:       &1                   \\
% WARN TAG-OPEN-MC:          &1                   \\
% WARN SHIPOUT-MC-OPEN:      &1                    \\
% WARN SHIPOUT-UPS:          &0 & shouldn't happen \\
% WARN TEX-MC-INSERT-MISSING:&0 & shouldn't happen \\
% WARN TEX-MC-INSERT-NO-KIDS:&2 & e.g. from empty hbox
% \end{tabular}
%
% \subsection{Info messages from the lua-code}
% The messages are triggered if the log-level is at least equal to the number.
% |TAG| messages are from the traversing function, |TEX| from code used in the tagpdf-mc module.
% |PARENTREE| is the code building the parenttree.
%
% \begin{longtable}{>{\ttfamily}lll}
% message & log-level & remark \\\hline\endhead
% INFO SHIPOUT-INSERT-LAST-EMC & 3 & finish of shipout code \\
% INFO SPACE-FUNCTION-FONT     & 3 & interwordspace code \\
% INFO TAG-ABSPAGE & 3  \\
% INFO TAG-ARGS & 4     \\
% INFO TAG-ENDHEAD & 4  \\
% INFO TAG-ENDHEAD & 4  \\
% INFO TAG-HEAD & 3     \\
% INFO TAG-INSERT-ARTIFACT & 3  \\
% INFO TAG-INSERT-BDC & 3  \\
% INFO TAG-INSERT-EMC & 3  \\
% INFO TAG-INSERT-TAG & 3  \\
% INFO TAG-KERN-SUBTYPE & 4\\
% INFO TAG-MATH-SUBTYPE & 4 \\
% INFO TAG-MC-COMPARE & 4   \\
% INFO TAG-MC-INTO-PAGE & 3 \\
% INFO TAG-NEW-MC-NODE & 4  \\
% INFO TAG-NODE & 3  \\
% INFO TAG-NO-HEAD & 3 \\
% INFO TAG-NOT-TAGGED & 2 & replaced by artifact\\
% INFO TAG-QUITTING-BOX & 4 \\
% INFO TAG-STORE-MC-KID & 4 \\
% INFO TAG-TRAVERSING-BOX 3 \\
% INFO TAG-USE-ACTUALTEXT & 3 \\
% INFO TAG-USE-ALT & 3 \\
% INFO TAG-USE-RAW & 3 \\
% INFO TEX-MC-INSERT-KID & 3 \\
% INFO TEX-MC-INSERT-KID-TEST & 4 \\
% INFO TEX-MC-INTO-STRUCT & 3 \\
% INFO TEX-STORE-MC-DATA & 3  \\
% INFO TEX-STORE-MC-KID &  3  \\
% INFO PARENTTREE-CHUNKS &  3 \\
% INFO PARENTTREE-NO-DATA & 3\\
% INFO PARENTTREE-NUM & 3 \\
% INFO PARENTTREE-NUMENTRY & 3 \\
% INFO PARENTTREE-STRUCT-OBJREF & 4
% \end{longtable}
%
% \subsection{Debug mode messages and code}
% If the package tagpdf-debug is loaded a number of commands are redefined and
% enhanced with additional commands which can be used to output debug messages or
% collect statistics. The commands are present but do nothing if the log-level is zero.
% \begin{tabular}{llll}
%   command             & name     & action   & remark\\\hline
%   |\tag_mc_begin:n|   & mc-begin-insert & msg   \\
%                       & mc-begin-ignore & msg     & if inactive \\
% \end{tabular}
%
% \subsection{Messages}
% \begin{function}{
%  mc-nested,
%  mc-tag-missing,
%  mc-label-unknown,
%  mc-used-twice,
%  mc-used-twice,
%  mc-not-open,
%  mc-pushed,
%  mc-popped,
%  mc-current}
%  Various messages related to mc-chunks. TODO document their meaning.
% \end{function}
% \begin{function}{struct-no-objnum,struct-faulty-nesting,
%  struct-missing-tag,struct-used-twice,struct-label-unknown,struct-show-closing}
% Various messages related to structure. TODO document their meaning.
% \end{function}
%
% \begin{function}{attr-unknown}
% Message if an attribute i sunknown.
% \end{function}
%
% \begin{function}{role-missing,role-unknown,role-unknown-tag,role-tag,new-tag}
% Messages related to role mapping.
% \end{function}
%
% \begin{function}{tree-mcid-index-wrong}
% Used in the tree code, typically indicates the document must be rerun.
% \end{function}
%
% \begin{function}{sys-no-interwordspace}
% Message if an engine doesn't support inter word spaces
% \end{function}
%
% \begin{function}{para-hook-count-wrong}
% Message if the number of begin paragraph and end paragraph differ. This normally means faulty structure.
% \end{function}
% \end{documentation}
% \begin{implementation}
%    \begin{macrocode}
%<@@=tag>
%<*header>
\ProvidesExplPackage {tagpdf-checks-code} {2023-04-24} {0.98f}
 {part of tagpdf - code related to checks, conditionals, debugging and messages}
%</header>
%    \end{macrocode}
% \section{Messages}
% \subsection{Messages related to mc-chunks}
% \begin{macro}{mc-nested}
% This message is issue is a mc is opened before the previous has been closed.
% This is not relevant for luamode, as the attributes don't care about this.
% It is used in the |\@@_check_mc_if_nested:| test.
%    \begin{macrocode}
%<*package>
\msg_new:nnn { tag } {mc-nested} { nested~marked~content~found~-~mcid~#1 }
%    \end{macrocode}
% \end{macro}
% \begin{macro}{mc-tag-missing}
% If the tag is missing
%    \begin{macrocode}
\msg_new:nnn { tag } {mc-tag-missing} { required~tag~missing~-~mcid~#1 }
%    \end{macrocode}
% \end{macro}
% \begin{macro}{mc-label-unknown}
% If the label of a mc that is used in another place is not known (yet)
% or has been undefined as the mc was already used.
%    \begin{macrocode}
\msg_new:nnn { tag } {mc-label-unknown}
  { label~#1~unknown~or~has~been~already~used.\\
    Either~rerun~or~remove~one~of~the~uses. }
%    \end{macrocode}
% \end{macro}
% \begin{macro}{mc-used-twice}
% An mc-chunk can be inserted only in one structure. This indicates wrong coding
% and so should at least give a warning.
%    \begin{macrocode}
\msg_new:nnn { tag } {mc-used-twice} { mc~#1~has~been~already~used }
%    \end{macrocode}
% \end{macro}
% \begin{macro}{mc-not-open}
% This is issued if a |\tag_mc_end:| is issued wrongly, wrong coding.
%    \begin{macrocode}
\msg_new:nnn { tag } {mc-not-open} { there~is~no~mc~to~end~at~#1 }
%    \end{macrocode}
% \end{macro}
% \begin{macro}{mc-pushed,mc-popped}
% Informational messages about mc-pushing.
%    \begin{macrocode}
\msg_new:nnn { tag } {mc-pushed} { #1~has~been~pushed~to~the~mc~stack}
\msg_new:nnn { tag } {mc-popped} { #1~has~been~removed~from~the~mc~stack }
%    \end{macrocode}
% \end{macro}
% \begin{macro}{mc-current}
% Informational messages about current mc state.
%    \begin{macrocode}
\msg_new:nnn { tag } {mc-current}
  { current~MC:~
    \bool_if:NTF\g_@@_in_mc_bool
      {abscnt=\@@_get_mc_abs_cnt:,~tag=\g_@@_mc_key_tag_tl}
      {no~MC~open,~current~abscnt=\@@_get_mc_abs_cnt:"}
  }
%    \end{macrocode}
% \end{macro}
% \subsection{Messages related to structures}
% \begin{macro}{struct-unknown}
% if for example a parent key value points to structure that doesn't exist (yet)
%    \begin{macrocode}
\msg_new:nnn { tag } {struct-unknown}
   { structure~with~number~#1~doesn't~exist\\ #2 }
%    \end{macrocode}
% \end{macro}
% \begin{macro}{struct-no-objnum}
% Should not happen \ldots
%    \begin{macrocode}
\msg_new:nnn { tag } {struct-no-objnum} { objnum~missing~for~structure~#1 }
%    \end{macrocode}
% \end{macro}
% \begin{macro}{struct-faulty-nesting}
% This indicates that there is somewhere one |\tag_struct_end:| too much.
% This should be normally an error.
%    \begin{macrocode}
\msg_new:nnn { tag }
  {struct-faulty-nesting}
  { there~is~no~open~structure~on~the~stack }
%    \end{macrocode}
% \end{macro}
% \begin{macro}{struct-missing-tag}
% A structure must have a tag.
%    \begin{macrocode}
\msg_new:nnn { tag } {struct-missing-tag} { a~structure~must~have~a~tag! }
%    \end{macrocode}
% \end{macro}
% \begin{macro}{struct-used-twice}
%    \begin{macrocode}
\msg_new:nnn { tag } {struct-used-twice}
  { structure~with~label~#1~has~already~been~used}
%    \end{macrocode}
% \end{macro}
% \begin{macro}{struct-label-unknown}
% label is unknown, typically needs a rerun.
%    \begin{macrocode}
\msg_new:nnn { tag } {struct-label-unknown}
  { structure~with~label~#1~is~unknown~rerun}
%    \end{macrocode}
% \end{macro}
% \begin{macro}{struct-show-closing}
% Informational message shown if log-mode is high enough
%    \begin{macrocode}
\msg_new:nnn { tag } {struct-show-closing}
  { closing~structure~#1~tagged~\use:e{\prop_item:cn{g_@@_struct_#1_prop}{S}} }
%    \end{macrocode}
% \end{macro}
% \begin{macro}{tree-struct-still-open}
% Message issued at the end if there are beside Root other
% open structures on the stack.
%    \begin{macrocode}
\msg_new:nnn { tag } {tree-struct-still-open}
  { 
    There~are~still~open~structures~on~the~stack!\\
    The~stack~contains~\seq_use:Nn\g_@@_struct_tag_stack_seq{,}.\\
    The~structures~are~automatically~closed,\\
    but~their~nesting~can~be~wrong.  
  }
%    \end{macrocode}
% \end{macro} 
%
% \subsection{Attributes}
% Not much yet, as attributes aren't used so much.
% \begin{macro}{attr-unknown}
%    \begin{macrocode}
\msg_new:nnn { tag } {attr-unknown}  { attribute~#1~is~unknown}
%    \end{macrocode}
% \end{macro}
%
% \subsection{Roles}
% \begin{macro}{role-missing,role-unknown,role-unknown-tag}
% Warning message if either the tag or the role is missing
%    \begin{macrocode}
\msg_new:nnn { tag } {role-missing}     { tag~#1~has~no~role~assigned  }
\msg_new:nnn { tag } {role-unknown}     { role~#1~is~not~known  }
\msg_new:nnn { tag } {role-unknown-tag} { tag~#1~is~not~known  }
%    \end{macrocode}
% \end{macro}
% \begin{macro}{role-parent-child}
%  This is info and warning message about the containment rules between child and
%  parent tags.
%    \begin{macrocode}
\msg_new:nnn { tag } {role-parent-child} 
  { Rule~'#1'~-->~'#2'~is~#3}
%    \end{macrocode}
% \end{macro}
% \begin{macro}{role-parent-child}
%  This is info and warning message about the containment rules between child and
%  parent tags.
%    \begin{macrocode}
\msg_new:nnn { tag } {role-remapping} 
  { remapping~tag~to~#1 }
%    \end{macrocode}
% \end{macro}
% \begin{macro}{role-tag,new-tag}
% Info messages.
%    \begin{macrocode}
\msg_new:nnn { tag } {role-tag}         { mapping~tag~#1~to~role~#2  }
\msg_new:nnn { tag } {new-tag}          { adding~new~tag~#1 }
\msg_new:nnn { tag } {read-namespace}   { reading~namespace~definitions~tagpdf-ns-#1.def }
\msg_new:nnn { tag } {namespace-missing}{ namespace~definitions~tagpdf-ns-#1.def~not~found }
\msg_new:nnn { tag } {namespace-unknown}{ namespace~#1~is~not~declared }
%    \end{macrocode}
% \end{macro}
%
% \subsection{Miscellaneous}
% \begin{macro}{tree-mcid-index-wrong}
% Used in the tree code, typically indicates the document must be rerun.
%    \begin{macrocode}
\msg_new:nnn { tag } {tree-mcid-index-wrong}
  {something~is~wrong~with~the~mcid--rerun}
%    \end{macrocode}
% \end{macro}
% \begin{macro}{sys-no-interwordspace}
% Currently only pdflatex and lualatex have some support for real spaces.
%    \begin{macrocode}
\msg_new:nnn { tag } {sys-no-interwordspace}
  {engine/output~mode~#1~doesn't~support~the~interword~spaces}
%    \end{macrocode}
% \end{macro}
%
% \begin{macro}{\@@_check_typeout_v:n}
% A simple logging function. By default is gobbles its argument, but
% the log-keys sets it to typeout.
%    \begin{macrocode}
\cs_set_eq:NN \@@_check_typeout_v:n \use_none:n
%    \end{macrocode}
% \end{macro}
%
% \begin{macro}{para-hook-count-wrong}
% At the end of the document we check if the count of para-begin and para-end is
% identical. If not we issue a warning: this is normally a coding error and
% and breaks the structure.
%    \begin{macrocode}
\msg_new:nnnn { tag } {para-hook-count-wrong}
  {The~number~of~automatic~begin~(#1)~and~end~(#2)~#3~para~hooks~differ!}
  {This~quite~probably~a~coding~error~and~the~structure~will~be~wrong!}
%</package>
%    \end{macrocode}
% \end{macro}
% \section{Retrieving data}
% \begin{macro}[EXP]{\tag_get:n}
% This retrieves some data.
% This is a generic command to retrieve data. Currently
% the only sensible values for the argument are |mc_tag|, |struct_tag|
% and |struct_num|.
%    \begin{macrocode}
%<base>\cs_new:Npn \tag_get:n #1   { \use:c {@@_get_data_#1: } }
%    \end{macrocode}
% \end{macro}
%
% \section{User conditionals}
% \begin{macro}[pTF]{\tag_if_active:}
% This is a test it tagging is active. This allows packages
% to add conditional code.
% The test is true if all booleans, the global and the two local one are true.
%
%    \begin{macrocode}
%<*base>
\prg_new_conditional:Npnn \tag_if_active: { p , T , TF, F }
  { \prg_return_false: }
%</base>
%<*package>
\prg_set_conditional:Npnn \tag_if_active: { p , T , TF, F }
  {
     \bool_lazy_all:nTF
       {
         {\g_@@_active_struct_bool}
         {\g_@@_active_mc_bool}
         {\g_@@_active_tree_bool}
         {\l_@@_active_struct_bool}
         {\l_@@_active_mc_bool}
       }
       {
         \prg_return_true:
       }
       {
         \prg_return_false:
       }
  }
%    \end{macrocode}
% \end{macro}
%
% \section{Internal checks}
% These are checks used in various places in the code.
%
% \subsection{checks for active tagging}
% \begin{macro}[TF]{\@@_check_if_active_mc:,\@@_check_if_active_struct:}
% This checks if mc are active.
%    \begin{macrocode}
\prg_new_conditional:Npnn \@@_check_if_active_mc: {T,F,TF}
  {
    \bool_lazy_and:nnTF { \g_@@_active_mc_bool } { \l_@@_active_mc_bool }
      {
         \prg_return_true:
      }
      {
         \prg_return_false:
      }
  }
\prg_new_conditional:Npnn \@@_check_if_active_struct: {T,F,TF}
  {
    \bool_lazy_and:nnTF { \g_@@_active_struct_bool } { \l_@@_active_struct_bool }
      {
         \prg_return_true:
      }
      {
         \prg_return_false:
      }
  }
%    \end{macrocode}
% \end{macro}
% \subsection{Checks related to structures}
% \begin{macro}{\@@_check_structure_has_tag:n}
% Structures must have a tag, so we check if the S entry is in the property.
% It is an error if this is missing. The argument is a number.
% The tests for existence and type is split in structures,
% as the tags are stored differently to the mc case.
%    \begin{macrocode}
\cs_new_protected:Npn \@@_check_structure_has_tag:n #1 %#1 struct num
  {
    \prop_if_in:cnF { g_@@_struct_#1_prop }
      {S}
      {
        \msg_error:nn { tag } {struct-missing-tag}
      }
  }
%    \end{macrocode}
% \end{macro}
% \begin{macro}{\@@_check_structure_tag:N}
% This checks if the name of the tag is known,
% either because it is a standard type or has been rolemapped.
%    \begin{macrocode}
\cs_new_protected:Npn \@@_check_structure_tag:N #1
  {
    \prop_if_in:NoF \g_@@_role_tags_NS_prop {#1}
      {
        \msg_warning:nnx { tag } {role-unknown-tag} {#1}
      }
  }
%    \end{macrocode}
% \end{macro}
% \begin{macro}{\@@_check_info_closing_struct:n}
% This info message is issued at a closing structure, the use should be guarded by
% log-level.
%    \begin{macrocode}
\cs_new_protected:Npn \@@_check_info_closing_struct:n #1 %#1 struct num
  {
    \int_compare:nNnT {\l_@@_loglevel_int} > { 0 }
      {
        \msg_info:nnn { tag } {struct-show-closing} {#1}
      }
  }

\cs_generate_variant:Nn \@@_check_info_closing_struct:n {o,x}
%    \end{macrocode}
% \end{macro}
% \begin{macro}{\@@_check_no_open_struct:}
% This checks if there is an open structure. It should be used when trying to
% close a structure. It errors if false.%
%    \begin{macrocode}
\cs_new_protected:Npn \@@_check_no_open_struct:
  {
    \msg_error:nn { tag } {struct-faulty-nesting}
  }
%    \end{macrocode}
% \end{macro}
% \begin{macro}{\@@_check_struct_used:n}
% This checks if a stashed structure has already been used.
%    \begin{macrocode}
\cs_new_protected:Npn \@@_check_struct_used:n #1 %#1 label
  {
    \prop_get:cnNT
      {g_@@_struct_\@@_ref_value:enn{tagpdfstruct-#1}{tagstruct}{unknown}_prop}
      {P}
      \l_@@_tmpa_tl
      {
        \msg_warning:nnn { tag } {struct-used-twice} {#1}
      }
  }
%    \end{macrocode}
% \end{macro}
%
% \subsection{Checks related to roles}
% \begin{macro}{\@@_check_add_tag_role:nn}
% This check is used when defining a new role mapping.
%    \begin{macrocode}
\cs_new_protected:Npn \@@_check_add_tag_role:nn #1 #2 %#1 tag, #2 role
  {
    \tl_if_empty:nTF {#2}
      {
        \msg_error:nnn { tag } {role-missing} {#1}
      }
      {
        \prop_get:NnNTF \g_@@_role_tags_NS_prop {#2} \l_tmpa_tl
          {
            \int_compare:nNnT {\l_@@_loglevel_int} > { 0 }
              {
                \msg_info:nnnn { tag } {role-tag} {#1} {#2}
              }
          }
          {
            \msg_error:nnn { tag } {role-unknown} {#2}
          }
      }
  }
%    \end{macrocode}
% Similar with a namespace
%    \begin{macrocode}
\cs_new_protected:Npn \@@_check_add_tag_role:nnn #1 #2 #3 %#1 tag/NS, #2 role #3 namespace
  {
    \tl_if_empty:nTF {#2}
      {
        \msg_error:nnn { tag } {role-missing} {#1}
      }
      {
        \prop_get:cnNTF { g_@@_role_NS_#3_prop } {#2} \l_tmpa_tl
          {
            \int_compare:nNnT {\l_@@_loglevel_int} > { 0 }
              {
                \msg_info:nnnn { tag } {role-tag} {#1} {#2/#3}
              }
          }
          {
            \msg_error:nnn { tag } {role-unknown} {#2/#3}
          }
      }
  }
%    \end{macrocode}
% \end{macro}

% \subsection{Check related to mc-chunks}
%
% \begin{macro}{\@@_check_mc_if_nested:,\@@_check_mc_if_open:}
% Two tests if a mc is currently open. One for the true (for begin
% code),  one for the false part (for end code).
%    \begin{macrocode}
\cs_new_protected:Npn \@@_check_mc_if_nested:
  {
    \@@_mc_if_in:T
      {
        \msg_warning:nnx { tag } {mc-nested} { \@@_get_mc_abs_cnt: }
      }
  }

\cs_new_protected:Npn \@@_check_mc_if_open:
  {
    \@@_mc_if_in:F
      {
        \msg_warning:nnx { tag } {mc-not-open} { \@@_get_mc_abs_cnt: }
      }
  }
%    \end{macrocode}
% \end{macro}
% \begin{macro}{\@@_check_mc_pushed_popped:nn}
% This creates an information message if mc's are pushed or popped.
% The first argument is a word (pushed or popped), the second the tag name.
% With larger log-level the stack is shown too.
%    \begin{macrocode}
\cs_new_protected:Npn \@@_check_mc_pushed_popped:nn #1 #2
  {
    \int_compare:nNnT
      { \l_@@_loglevel_int } ={ 2 }
      { \msg_info:nnx {tag}{mc-#1}{#2} }
    \int_compare:nNnT
      { \l_@@_loglevel_int } > { 2 }
      {
        \msg_info:nnx {tag}{mc-#1}{#2}
        \seq_log:N \g_@@_mc_stack_seq
      }
  }
%    \end{macrocode}
% \end{macro}
% \begin{macro}{\@@_check_mc_tag:N}
% This checks if the mc has a (known) tag.
%    \begin{macrocode}
\cs_new_protected:Npn \@@_check_mc_tag:N #1  %#1 is var with a tag name in it
  {
    \tl_if_empty:NT #1
      {
        \msg_error:nnx { tag } {mc-tag-missing} { \@@_get_mc_abs_cnt: }
      }
   \prop_if_in:NoF \g_@@_role_tags_NS_prop {#1}
     {
       \msg_warning:nnx { tag } {role-unknown-tag} {#1}
     }
  }
%    \end{macrocode}
% \end{macro}
%
% \begin{variable}{\g_@@_check_mc_used_intarray, \@@_check_init_mc_used:}
% This variable holds the list of used mc numbers.
% Everytime we store a mc-number we will add one the relevant array index
% If everything is right at the end there should be only 1 until the max count
% of the mcid. 2 indicates that one mcid was used twice, 0 that we lost one.
% In engines other than luatex the total number of all intarray entries
% are restricted so we use only  a rather small value of 65536, and we initialize
% the array only at first used, guarded by the log-level.
% This check is probably only needed for debugging.
% TODO does this really make sense to check? When can it happen??
%    \begin{macrocode}
\cs_new_protected:Npn \@@_check_init_mc_used:
  {
    \intarray_new:Nn \g_@@_check_mc_used_intarray { 65536 }
    \cs_gset_eq:NN \@@_check_init_mc_used: \prg_do_nothing:
  }
%    \end{macrocode}
% \end{variable}
%
% \begin{macro}{\@@_check_mc_used:n}
% This checks if a mc is used twice.
%    \begin{macrocode}
\cs_new_protected:Npn \@@_check_mc_used:n #1 %#1 mcid abscnt
  {
    \int_compare:nNnT {\l_@@_loglevel_int} > { 2 }
      {
        \@@_check_init_mc_used:
        \intarray_gset:Nnn \g_@@_check_mc_used_intarray
          {#1}
          { \intarray_item:Nn \g_@@_check_mc_used_intarray {#1} + 1 }
        \int_compare:nNnT
          {
            \intarray_item:Nn \g_@@_check_mc_used_intarray {#1}
          }
          >
          { 1 }
          {
            \msg_warning:nnn { tag } {mc-used-twice} {#1}
          }
      }
  }
%    \end{macrocode}
% \end{macro}
% \begin{macro}{\@@_check_show_MCID_by_page:}
% This allows to show the mc on a page. Currently unused.
%    \begin{macrocode}
\cs_new_protected:Npn \@@_check_show_MCID_by_page:
  {
    \tl_set:Nx \l_@@_tmpa_tl
      {
        \@@_ref_value_lastpage:nn
          {abspage}
          {-1}
      }
    \int_step_inline:nnnn {1}{1}
      {
        \l_@@_tmpa_tl
      }
      {
        \seq_clear:N \l_tmpa_seq
        \int_step_inline:nnnn
          {1}
          {1}
          {
            \@@_ref_value_lastpage:nn
              {tagmcabs}
              {-1}
          }
          {
            \int_compare:nT
              {
                \@@_ref_value:enn
                  {mcid-####1}
                  {tagabspage}
                  {-1}
                =
                ##1
             }
             {
               \seq_gput_right:Nx \l_tmpa_seq
                 {
                   Page##1-####1-
                   \@@_ref_value:enn
                     {mcid-####1}
                     {tagmcid}
                     {-1}
                 }
             }
          }
          \seq_show:N \l_tmpa_seq
      }
  }
%    \end{macrocode}
% \end{macro}
%
% \subsection{Checks related to the state of MC on a page or in a split stream}
% The following checks are currently only usable in generic mode as they
% rely on the marks defined in the mc-generic module. They are used to detect
% if a mc-chunk has been split by a page break or similar and additional
% end/begin commands are needed.
%
% \begin{macro}[pTF]{\@@_check_mc_in_galley:}
% At first we need a test to decide if |\tag_mc_begin:n| (tmb) and |\tag_mc_end:|
% (tme) has been used at all on the current galley. As each command issues
% two slightly different marks we can do it by comparing firstmarks and botmarks.
% The test assumes that the marks have been already mapped into the sequence with
% |\@@_mc_get_marks:|. As |\seq_if_eq:NNTF| doesn't exist we use the tl-test.
%    \begin{macrocode}
\prg_new_conditional:Npnn \@@_check_if_mc_in_galley: { T,F,TF }
 {
   \tl_if_eq:NNTF \l_@@_mc_firstmarks_seq \l_@@_mc_botmarks_seq
    { \prg_return_false: }
    { \prg_return_true: }
 }
%    \end{macrocode}
% \end{macro}

% \begin{macro}[pTF]{\@@_check_if_mc_tmb_missing:}
% This checks if a extra top mark (\enquote{extra-tmb}) is needed.
% According to the analysis this the case if the firstmarks start with
% |e-| or |b+|.
% Like above we assume that the marks content is already in the seq's.
%    \begin{macrocode}
\prg_new_conditional:Npnn \@@_check_if_mc_tmb_missing: { T,F,TF }
 {
  \bool_if:nTF
    {
      \str_if_eq_p:ee {\seq_item:Nn \l_@@_mc_firstmarks_seq {1}}{e-}
      ||
      \str_if_eq_p:ee {\seq_item:Nn \l_@@_mc_firstmarks_seq {1}}{b+}
    }
    { \prg_return_true: }
    { \prg_return_false: }
 }
%    \end{macrocode}
% \end{macro}

% \begin{macro}[pTF]{\@@_check_if_mc_tme_missing:}
% This checks if a extra bottom mark (\enquote{extra-tme}) is needed.
% According to the analysis this the case if the botmarks starts with
% |b+|.
% Like above we assume that the marks content is already in the seq's.
%    \begin{macrocode}
\prg_new_conditional:Npnn \@@_check_if_mc_tme_missing: { T,F,TF }
 {
   \str_if_eq:eeTF {\seq_item:Nn \l_@@_mc_botmarks_seq {1}}{b+}
    { \prg_return_true: }
    { \prg_return_false: }
 }
%    \end{macrocode}
% \end{macro}
%    \begin{macrocode}
%</package>
%    \end{macrocode}
%    \begin{macrocode}
%<*debug>
%    \end{macrocode}
% Code for tagpdf-debug. This will probably change over time.
% At first something for the mc commands.
%    \begin{macrocode}
\msg_new:nnn { tag / debug } {mc-begin} { MC~begin~#1~with~options:~\tl_to_str:n{#2}~[\msg_line_context:] }
\msg_new:nnn { tag / debug } {mc-end}   { MC~end~#1~[\msg_line_context:] }

\cs_new_protected:Npn \@@_debug_mc_begin_insert:n #1
 {
   \int_compare:nNnT { \l_@@_loglevel_int } > {0}
     {
        \msg_note:nnnn { tag / debug } {mc-begin} {inserted} { #1 }
     }
 }
\cs_new_protected:Npn \@@_debug_mc_begin_ignore:n #1
 {
   \int_compare:nNnT { \l_@@_loglevel_int } > {0}
     {
        \msg_note:nnnn { tag / debug } {mc-begin } {ignored} { #1 }
     }
 }
\cs_new_protected:Npn \@@_debug_mc_end_insert:
 {
   \int_compare:nNnT { \l_@@_loglevel_int } > {0}
     {
        \msg_note:nnn { tag / debug } {mc-end} {inserted}
     }
 }
\cs_new_protected:Npn \@@_debug_mc_end_ignore:
 {
   \int_compare:nNnT { \l_@@_loglevel_int } > {0}
     {
        \msg_note:nnn { tag / debug } {mc-end } {ignored}
     }
 }
%    \end{macrocode}
% And now something for the structures
%    \begin{macrocode}
\msg_new:nnn { tag / debug } {struct-begin}
  {
    Struct~\tag_get:n{struct_num}~begin~#1~with~options:~\tl_to_str:n{#2}~[\msg_line_context:]
  }
\msg_new:nnn { tag / debug } {struct-end}
  {
    Struct~end~#1~[\msg_line_context:]
  }
\msg_new:nnn { tag / debug } {struct-end-wrong}
  {
    Struct~end~'#1'~doesn't~fit~start~'#2'~[\msg_line_context:]
  }

\cs_new_protected:Npn \@@_debug_struct_begin_insert:n #1
 {
   \int_compare:nNnT { \l_@@_loglevel_int } > {0}
     {
        \msg_note:nnnn { tag / debug } {struct-begin} {inserted} { #1 }
        \seq_log:N \g_@@_struct_tag_stack_seq
     }
 }
\cs_new_protected:Npn \@@_debug_struct_begin_ignore:n #1
 {
   \int_compare:nNnT { \l_@@_loglevel_int } > {0}
     {
        \msg_note:nnnn { tag / debug } {struct-begin } {ignored} { #1 }
     }
 }
\cs_new_protected:Npn \@@_debug_struct_end_insert:
 {
   \int_compare:nNnT { \l_@@_loglevel_int } > {0}
     {
        \msg_note:nnn { tag / debug } {struct-end} {inserted}
        \seq_log:N \g_@@_struct_tag_stack_seq
     }
 }
\cs_new_protected:Npn \@@_debug_struct_end_ignore:
 {
   \int_compare:nNnT { \l_@@_loglevel_int } > {0}
     {
        \msg_note:nnn { tag / debug } {struct-end } {ignored}
     }
 }
\cs_new_protected:Npn \@@_debug_struct_end_check:n #1
 {
   \int_compare:nNnT { \l_@@_loglevel_int } > {0}
    {
      \seq_get:NNT \g_@@_struct_tag_stack_seq \l_@@_tmpa_tl
        {
          \str_if_eq:eeF  
           {#1}
           {\exp_last_unbraced:NV\use_i:nn \l_@@_tmpa_tl}      
           {
             \msg_warning:nnxx { tag/debug }{ struct-end-wrong } 
              {#1}
              {\exp_last_unbraced:NV\use_i:nn \l_@@_tmpa_tl}
           }                    
        }
     }
 }
 
%    \end{macrocode}
%    \begin{macrocode}
%</debug>
%    \end{macrocode}
% \end{implementation}
% \PrintIndex

\DocumentMetadata{uncompress,testphase=phase-II,pdfversion=2.0}
\documentclass{article}

\begin{document}

Remapping tags. 

In some context it can be necessary to remap the tags. 

That means instead of tag=H1 or tag=section one wants the effect of tag=Span.
With tags set by internal code, one could use tag=somecommand and then redefine somecommand,
but that wouldn't work with structures added manually by the user. 

We define a prop tag->new tag

\ExplSyntaxOn
%\prop_show:c{g__tag_role_NS_pdf_prop}
%\prop_show:c{g__tag_role_NS_pdf2_prop}
%\prop_show:c{g__tag_role_NS_mathml_prop}
%\prop_show:c{g__tag_role_NS_latex_prop}
%\prop_show:c{g__tag_role_NS_latex-book_prop}
%\prop_new:N\g__tag_role_remap_inline_prop
%\prop_gset_from_keyval:Nn\g__tag_role_remap_inline_prop{H1=P/pdf2,chapter=P/pdf}
%\pdf_version_compare:NnTF <{2.0}
% {
%   \cs_new_protected:Npn \__tag_role_remap_inline:nN #1#2
%    {
%      \prop_get:cnNF { g__tag_role_NS_latex-inline_prop }{#1}#2
%       {\tl_set:Nn#2{#1}}
%    }
% }
% {
%   \cs_new_protected:Npn \__tag_role_remap_inline:nN #1#2
%    {
%      \prop_get:cnNTF { g__tag_role_NS_latex-inline_prop }{#1}#2
%       {\tl_set:Nn#2{#1/latex-inline}}
%       {\tl_set:Nn#2{#1}}
%    }
% 
% }   
%\cs_new_protected:Npn \__tag_role_remap_id:nN #1#2
% { \tl_set:Nn#2{#1} } 

\cs_set_eq:NN\__tag_role_remap_default:nN \__tag_role_remap_id:nN
 
\cs_set_eq:NN\__tag_role_remap: \__tag_role_remap_inline: 

\tl_set:Nn \l__tag_role_remap_tag_tl {chapter} 

\tl_set:Nn \l__tag_role_remap_NS_tl {latex} 

\__tag_role_remap:

\tl_show:N\l__tag_role_remap_tag_tl
%\__tag_role_remap_default:nN{chapter}\l_tmpa_tl 

abb 
\tagmcend\tagstructbegin{tag=H1}\tagmcbegin{tag=P}blub\tagmcend\tagstructend\tagmcbegin{tag=P} 
ccc 

 
\ExplSyntaxOff

 

Hello.
\end{document}

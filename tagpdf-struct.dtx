% \iffalse meta-comment
%
%% File: tagpdf-struct.dtx
%
% Copyright (C) 2019-2022 Ulrike Fischer
%
% It may be distributed and/or modified under the conditions of the
% LaTeX Project Public License (LPPL), either version 1.3c of this
% license or (at your option) any later version.  The latest version
% of this license is in the file
%
%    https://www.latex-project.org/lppl.txt
%
% This file is part of the "tagpdf bundle" (The Work in LPPL)
% and all files in that bundle must be distributed together.
%
% -----------------------------------------------------------------------
%
% The development version of the bundle can be found at
%
%    https://github.com/u-fischer/tagpdf
%
% for those people who are interested.
%<*driver>
\DocumentMetadata{}
\documentclass{l3doc}
\usepackage{array,booktabs,caption}
\hypersetup{pdfauthor=Ulrike Fischer,
 pdftitle=tagpdf-tree module (tagpdf)}
\begin{document}
  \DocInput{\jobname.dtx}
\end{document}
%</driver>
% \fi
% \title{^^A
%   The \pkg{tagpdf-struct} module\\ Commands to create the structure   ^^A
%   \\ Part of the tagpdf package
% }
%
% \author{^^A
%  Ulrike Fischer\thanks
%    {^^A
%      E-mail:
%        \href{mailto:fischer@troubleshooting-tex.de}
%          {fischer@troubleshooting-tex.de}^^A
%    }^^A
% }
%
% \date{Version 0.97, released 2022-08-24}
% \maketitle
% \begin{documentation}
% \section{Public Commands}
% \begin{function}{\tag_struct_begin:n,\tag_struct_end:}
%   \begin{syntax}
%     \cs{tag_struct_begin:n}\Arg{key-values}\\
%     \cs{tag_struct_end:}
%   \end{syntax}
%  These commands start and end a new structure.
%  They don't start a group. They set all their values globally.
% \end{function}
%  \begin{function}{\tag_struct_use:n}
%   \begin{syntax}
%     \cs{tag_struct_use:n}\Arg{label}
%   \end{syntax}
% These commands insert a structure previously stashed away as kid
% into the currently active structure.
% A structure should be used only once,
% if the structure already has a parent a warning is issued.
% \end{function}
%  \begin{function}{\tag_struct_object_ref:n,\tag_struct_object_ref:e}
%   \begin{syntax}
%     \cs{tag_struct_object_ref:n}\Arg{struct number}
%   \end{syntax}
%   This is a small wrapper around |\pdf_object_ref:n| to retrieve the
%   object reference of the structure with the number \meta{struct number}.
%   This number can be retrieved and stored for the current structure
%   for example with \cs{tag_get:n}\Arg{struct_num}. Be aware that it can only
%   be used if the structure has already been created and that it doesn't check
%   if the object actually exists!
%  \end{function}
%
% The following two functions are used to add annotations. They must be used
% together and with care to get the same numbers. Perhaps some improvements are needed
% here.
%  \begin{function}{\tag_struct_insert_annot:nn}
%   \begin{syntax}
%     \cs{tag_struct_insert_annot:nn}\Arg{object reference}\Arg{struct parent number}
%   \end{syntax}
% This inserts an annotation in the structure. \meta{object reference}
% is there reference to the annotation. \meta{struct parent number}
% should be the same number as had been inserted with \cs{tag_struct_parent_int:}
% as |StructParent| value to the dictionary of the annotion.
% The command will increase the value of the counter
% used by \cs{tag_struct_parent_int:}.
% \end{function}
% \begin{function}{\tag_struct_parent_int:}
%   \begin{syntax}
%     \cs{tag_struct_parent_int:}
%   \end{syntax}
% This gives back the next free /StructParent number (assuming that it is
% together with \cs{tag_struct_insert_annot:nn} which will increase the number.
% \end{function}
% \section{Public keys}
% \subsection{Keys for the structure commands}
% \begin{function}{tag (struct-key)}
% This is required. The value of the key is normally one of the
% standard types listed in the main tagpdf documentation.
% It is possible to setup new tags/types.
% The value can also be of the form |type/NS|, where |NS| is the
% shorthand of a declared name space.
% Currently the names spaces |pdf|, |pdf2|, |mathml| and |user| are defined.
% This allows to use a different name space than
% the one connected by default to the tag. But normally this should not be needed.
% \end{function}
% \begin{function}{stash (struct-key)}
%  Normally a new structure inserts itself as a kid
%  into the currently active structure. This key prohibits this.
%  The structure is nevertheless from now on
%  \enquote{the current active structure}
%  and parent for following  marked content and structures.
% \end{function}
% \begin{function}{label (struct-key)}
% This key sets a label by which
% one can refer to the structure. It is e.g.
% used by \cs{tag_struct_use:n} (where a real label is actually not
% needed as you can only use structures already defined), and by the
% |ref| key (which can refer to future structures).
% Internally the label name will start with \texttt{tagpdfstruct-} and it stores
% the two attributs |tagstruct| (the structure number) and |tagstructobj| (the
% object reference).
% \end{function}
% \begin{function}{parent (struct-key)}
% By default a structure is added as kid to the currently active structure.
% With the parent key one can choose another parent. The value is a structure number which
% must refer to an already existing, previously created structure. Such a structure
% number can for example be have been stored with \cs{tag_get:n}, but one can also use
% a label on the parent structure and then use
% \cs{ref_value:nn}|{tagpdfstruct-label}{tagstruct}| to retrieve it.
% \end{function}
% \begin{function}{title (struct-key),title-o (struct-key)}
% This keys allows to set the dictionary entry
% \texttt{/Title} in the structure object.
% The value is handled as verbatim string and hex encoded.
% Commands are not expanded. |title-o| will expand the value once.
% \end{function}
%
% \begin{function}{alt (struct-key)}
% This key inserts an \texttt{/Alt} value in the dictionary of structure object.
% The value is handled as verbatim string and hex encoded.
% The value will be expanded first once.
% \end{function}
% \begin{function}{actualtext (struct-key)}
% This key inserts an \texttt{/ActualText} value in the dictionary of structure object.
% The value is handled as verbatim string and hex encoded.
% The value will be expanded first once.
% \end{function}
% \begin{function}{lang (struct-key)}
% This key allows to set the language for a structure element. The value should be a bcp-identifier,
% e.g. |de-De|.
% \end{function}
% \begin{function}{ref (struct-key)}
%  This key allows to add references to other structure elements,
%  it adds the |/Ref| array to the structure.
% The value should be a comma separated list of structure labels
% set with the |label| key. e.g. |ref={label1,label2}|.
% \end{function}
% \begin{function}{E (struct-key)}
%  This key sets the |/E| key, the expanded form of an
%  abbreviation or an acronym
%  (I couldn't think of a better name, so I sticked to E).
% \end{function}
% \begin{function}{AF (struct-key),AFinline (struct-key),AFinline-o (struct-key)}
% \begin{syntax}
% AF = \meta{object name}\\
% AF-inline = \meta{text content}\\
% \end{syntax}
% These keys allows to reference an associated file in the structure element.
% The value \meta{object name} should be the name of an object pointing
% to the \texttt{/Filespec} dictionary as expected by
% |\pdf_object_ref:n| from a current \texttt{l3kernel}.
%
% The value |AF-inline| is some text,
% which is embedded in the PDF as a text file with mime type text/plain.
% |AF-inline-o| is like |AF-inline| but expands the value once.
%
% Future versions will perhaps extend this to more mime types, but it is
% still a research task to find out what is really needed.
%
% |AF| can be used more than once, to associate more than one file. The inline
% keys can be used only once per structure. Additional calls are ignored.
% \end{function}
%
% \begin{function}{attribute (struct-key)}
%  This key takes as argument a comma list of attribute names
%  (use braces to protect the commas from the external key-val parser)
%  and allows to add one or more attribute dictionary entries in
%  the structure object. As an example
%     \begin{verbatim}
%      \tagstructbegin{tag=TH,attribute= TH-row}
%      \end{verbatim}
%  Attribute names and their content must be declared first in \cs{tagpdfsetup}.
%
%  \end{function}
%
% \begin{function}{attribute-class (struct-key)}
%  This key takes as argument a comma list of attribute class names
%  (use braces to protect the commas from the external key-val parser)
%  and allows to add one or more attribute classes to the structure object.
%
%  Attribute class names and their content
%  must be declared first in \cs{tagpdfsetup}.
% \end{function}
% \subsection{Setup keys}
% \begin{function}{newattribute (setup-key)}
% \begin{syntax}
% newattribute = \Arg{name}\Arg{Content}
% \end{syntax}
% This key can be used in the setup command \cs{tagpdfsetup} and allow to declare a
% new attribute, which can be used as attribute or attribute class.
% The value are two brace groups, the first contains the name, the second the content.
% \begin{verbatim}
% \tagpdfsetup
%  {
%   newattribute =
%    {TH-col}{/O /Table /Scope /Column},
%   newattribute =
%    {TH-row}{/O /Table /Scope /Row},
%   }
% \end{verbatim}
%
% \end{function}
% \begin{function}{root-AF (setup-key)}
% \begin{syntax}
% root-AF = \meta{object name}
% \end{syntax}
% This key can be used in the setup command \cs{tagpdfsetup} and allows
% to add associated files to the root structure. Like |AF| it can be used more than
% once to add more than one file.
% \end{function}
% \end{documentation}
% \begin{implementation}
%    \begin{macrocode}
%<@@=tag>
%<*header>
\ProvidesExplPackage {tagpdf-struct-code} {2022-08-24} {0.97}
 {part of tagpdf - code related to storing structure}
%</header>
%    \end{macrocode}
% \section{Variables}
% \begin{variable}{\c@g_@@_struct_abs_int}
% Every structure will have a unique, absolute number.
% I will use a latex counter for the structure count
% to have a chance to avoid double structures in align etc.
%
%    \begin{macrocode}
%<base>\newcounter  { g_@@_struct_abs_int }
%<base>\int_gzero:N \c@g_@@_struct_abs_int
%    \end{macrocode}
% \end{variable}
%
% \begin{variable}{\g_@@_struct_objR_seq}
% a sequence to store mapping between the
% structure number and the object number.
% We assume that structure numbers are assign
% consecutively and so the index of the seq can be used.
% A seq allows easy mapping over the structures.
%    \begin{macrocode}
%<*package>
\@@_seq_new:N  \g_@@_struct_objR_seq
%    \end{macrocode}
% \end{variable}

% \begin{variable}{\g_@@_struct_cont_mc_prop}
% in generic mode it can happen after
% a page break that we have to inject into a structure
% sequence an additional mc after. We will store this additional
% info in a property. The key is the absolut mc num, the value the pdf directory.
%    \begin{macrocode}
\@@_prop_new:N  \g_@@_struct_cont_mc_prop
%    \end{macrocode}
% \end{variable}
%
% \begin{variable}{\g_@@_struct_stack_seq}
% A stack sequence for the structure stack.
% When a sequence is opened it's number is put on the stack.
%    \begin{macrocode}
\seq_new:N    \g_@@_struct_stack_seq
\seq_gpush:Nn \g_@@_struct_stack_seq {0}
%    \end{macrocode}
% \end{variable}
%
% \begin{variable}{\g_@@_struct_tag_stack_seq}
% We will perhaps also need the tags. While it is possible to get them from the
% numbered stack, lets build a tag stack too.
%    \begin{macrocode}
\seq_new:N    \g_@@_struct_tag_stack_seq
\seq_gpush:Nn \g_@@_struct_tag_stack_seq {Root}
%    \end{macrocode}
% \end{variable}
%
% \begin{variable}{\g_@@_struct_stack_current_tl,\l_@@_struct_stack_parent_tmpa_tl}
% The global variable will hold the current structure number. It is already
% defined in \texttt{tagpdf-base}.
% The local temporary variable will hold the parent when we fetch it from the stack.
%    \begin{macrocode}
%</package>
%<base>\tl_new:N  \g_@@_struct_stack_current_tl
%<base>\tl_gset:Nn \g_@@_struct_stack_current_tl {\int_use:N\c@g_@@_struct_abs_int}
%<*package>
\tl_new:N     \l_@@_struct_stack_parent_tmpa_tl
%    \end{macrocode}
% \end{variable}
%
% I will need at least one structure: the StructTreeRoot
% normally it should have only one kid, e.g. the document element.

% The data of the StructTreeRoot and the StructElem are in properties:
% |\g_@@_struct_0_prop| for the root and
% |\g_@@_struct_N_prop|, $N \geq =1$ for the other.
%
% This creates quite a number of properties, so perhaps we will have to
% do this more efficiently in the future.
%
% All properties have at least the keys
% \begin{description}
%    \item[Type] StructTreeRoot or StructElem
%  \end{description}
% and the keys from the two following lists
% (the root has a special set of properties).
% the values of the prop should be already escaped properly
% when the entries are created (title,lange,alt,E,actualtext)
% \begin{variable}
%   {
%     \c_@@_struct_StructTreeRoot_entries_seq,
%     \c_@@_struct_StructElem_entries_seq
%   }
%  These seq contain the keys we support in the two object types.
%  They are currently no longer used, but are provided as documentation and
%  for potential future checks.
%  They should be adapted if there are changes in the PDF format.
%    \begin{macrocode}
\seq_const_from_clist:Nn \c_@@_struct_StructTreeRoot_entries_seq
  {%p. 857/858
    Type,              % always /StructTreeRoot
    K,                 % kid, dictionary or array of dictionaries
    IDTree,            % currently unused
    ParentTree,        % required,obj ref to the parent tree
    ParentTreeNextKey, % optional
    RoleMap,
    ClassMap,
    Namespaces,
    AF                 %pdf 2.0
  }

\seq_const_from_clist:Nn \c_@@_struct_StructElem_entries_seq
  {%p 858 f
    Type,              %always /StructElem
    S,                 %tag/type
    P,                 %parent
    ID,                %optional
    Ref,               %optional, pdf 2.0 Use?
    Pg,                %obj num of starting page, optional
    K,                 %kids
    A,                 %attributes, probably unused
    C,                 %class ""
    %R,                %attribute revision number, irrelevant for us as we
                       % don't update/change existing PDF and (probably)
                       % deprecated in PDF 2.0
    T,                 %title, value in () or <>
    Lang,              %language
    Alt,               % value in () or <>
    E,                 % abreviation
    ActualText,
    AF,                 %pdf 2.0, array of dict, associated files
    NS,                 %pdf 2.0, dict, namespace
    PhoneticAlphabet,   %pdf 2.0
    Phoneme             %pdf 2.0
  }
%    \end{macrocode}
% \end{variable}
%
% \subsection{Variables used by the keys}
% \begin{variable}{\g_@@_struct_tag_tl,\g_@@_struct_tag_NS_tl}
% Use by the tag key to store the tag and the namespace.
%    \begin{macrocode}
\tl_new:N \g_@@_struct_tag_tl
\tl_new:N \g_@@_struct_tag_NS_tl
%    \end{macrocode}
% \end{variable}
% \begin{variable}{\l_@@_struct_key_label_tl}
% This will hold the label value.
%    \begin{macrocode}
\tl_new:N \l_@@_struct_key_label_tl
%    \end{macrocode}
% \end{variable}
% \begin{variable}{\l_@@_struct_elem_stash_bool}
% This will keep track of the stash status
%    \begin{macrocode}
\bool_new:N \l_@@_struct_elem_stash_bool
%    \end{macrocode}
% \end{variable}
% \section{Commands}
%
% The properties must be in some places handled expandably.
% So I need an output handler for each prop, to get expandable output
% see \url{https://tex.stackexchange.com/questions/424208}.
% There is probably room here for a more efficient implementation.
% TODO check if this can now be implemented with the pdfdict commands.
% The property contains currently non pdf keys, but e.g. object numbers are
% perhaps no longer needed as we have named object anyway.
%
% \begin{macro}{\@@_struct_output_prop_aux:nn,\@@_new_output_prop_handler:n}
%    \begin{macrocode}
\cs_new:Npn \@@_struct_output_prop_aux:nn #1 #2 %#1 num, #2 key
  {
    \prop_if_in:cnT
      { g_@@_struct_#1_prop }
      { #2 }
      {
        \c_space_tl/#2~ \prop_item:cn{ g_@@_struct_#1_prop } { #2 }
      }
  }

\cs_new_protected:Npn \@@_new_output_prop_handler:n #1
  {
    \cs_new:cn { @@_struct_output_prop_#1:n }
      {
        \@@_struct_output_prop_aux:nn {#1}{##1}
      }
  }
%    \end{macrocode}
% \end{macro}
%
% \subsection{Initialization of the StructTreeRoot}
% The first structure element, the StructTreeRoot is special, so
% created manually. The underlying object is |@@/struct/0| which is currently
% created in the tree code (TODO move it here).
% The |ParentTree| and |RoleMap| entries are added at begin document
% in the tree code as they refer to object which are setup in other parts of the
% code. This avoid timing issues.
%
%    \begin{macrocode}
\tl_gset:Nn \g_@@_struct_stack_current_tl {0}
%    \end{macrocode}

%
% \begin{variable}{g_@@_struct_0_prop,g_@@_struct_kids_0_seq}
%    \begin{macrocode}
\@@_prop_new:c { g_@@_struct_0_prop }
\@@_new_output_prop_handler:n {0}
\@@_seq_new:c  { g_@@_struct_kids_0_seq }

\@@_prop_gput:cnn
  { g_@@_struct_0_prop }
  { Type }
  { /StructTreeRoot }



%    \end{macrocode}
% Namespaces are pdf 2.0 but it doesn't harm
% to have an empty entry. We could add a test, but if the code moves into
% the kernel, timing could get tricky.
%    \begin{macrocode}
\@@_prop_gput:cnx
  { g_@@_struct_0_prop }
  { Namespaces }
  { \pdf_object_ref:n { @@/tree/namespaces } }
%    \end{macrocode}
% \end{variable}
% 
% \subsection{Filling in the tag info}
% \begin{macro}{\@@_struct_set_tag_info:nnn }
% This adds or updates the tag info to a structure given by a number. 
% We need also the original data, so we store both.
%    \begin{macrocode}
\cs_new:Npn \@@_use_arg:nn #1#2{#1}

\pdf_version_compare:NnTF < {2.0}
 {
   \cs_new_protected:Npn \@@_struct_set_tag_info:nnn #1 #2 #3
     %#1 structure number, #2 tag
     { 
       \@@_prop_gput:cnx
         { g_@@_struct_#1_prop }         
         { S }
         { \exp_not:N\@@_use_arg:nn{\pdf_name_from_unicode_e:n{ #2 }}{#2}  } %
     }
 }     
 {     
   \cs_new_protected:Npn \@@_struct_set_tag_info:nnn #1 #2 #3
     {  
       \@@_prop_gput:cnx
         { g_@@_struct_#1_prop }
         { S }
         { \exp_not:N\@@_use_arg:nn{\pdf_name_from_unicode_e:n{ #2 }}{#2} } %
       \prop_get:NnNT \g_@@_role_NS_prop{#3} \l_@@_tmpa_tl
         {
           \@@_prop_gput:cnx
             { g_@@_struct_#1_prop }
             { NS }
             { \exp_not:N\@@_use_arg:nn{\l_@@_tmpa_tl}{#2}  } %
         }
     } 
 }     
\cs_generate_variant:Nn \@@_struct_set_tag_info:nnn {nVV}
%    \end{macrocode}
% \end{macro}
% \subsection{Handlings kids}
% Commands to store the kids. Kids in a structure can be a reference to a mc-chunk,
% an object reference to another structure element, or a object reference to an
% annotation (through an OBJR object).
% \begin{macro}{\@@_struct_kid_mc_gput_right:nn,\@@_struct_kid_mc_gput_right:nx}
% The command to store an mc-chunk, this is a dictionary of type MCR.
% It would be possible to write out the content directly as unnamed object
% and to store only the object reference, but probably this would be slower,
% and the PDF is more readable like this.
% The code doesn't try to avoid the use of the /Pg key by checking page numbers.
% That imho only slows down without much gain.
% In generic mode the page break code will perhaps to have to insert
% an additional mcid after an existing one. For this we use a property list
% At first an auxiliary to write the MCID dict. This should normally be expanded!
%    \begin{macrocode}
\cs_new:Npn \@@_struct_mcid_dict:n #1 %#1 MCID absnum
  {
     <<
      /Type \c_space_tl /MCR \c_space_tl
      /Pg
        \c_space_tl
      \pdf_pageobject_ref:n { \@@_ref_value:enn{mcid-#1}{tagabspage}{1} }
       /MCID \c_space_tl \@@_ref_value:enn{mcid-#1}{tagmcid}{1}
     >>
  }
%    \end{macrocode}
%    \begin{macrocode}
\cs_new_protected:Npn \@@_struct_kid_mc_gput_right:nn #1 #2 %#1 structure num, #2 MCID absnum%
  {
    \@@_seq_gput_right:cx
      { g_@@_struct_kids_#1_seq }
      {
        \@@_struct_mcid_dict:n {#2}
      }
    \@@_seq_gput_right:cn
      { g_@@_struct_kids_#1_seq }
      {
        \prop_item:Nn \g_@@_struct_cont_mc_prop {#2}
      }
  }
\cs_generate_variant:Nn \@@_struct_kid_mc_gput_right:nn {nx}

%    \end{macrocode}
% \end{macro}
%  \begin{macro}
%    {
%      \@@_struct_kid_struct_gput_right:nn,\@@_struct_kid_struct_gput_right:xx
%    }
%  This commands adds a structure as kid. We only need to record the object
%  reference in the sequence.
%    \begin{macrocode}
\cs_new_protected:Npn\@@_struct_kid_struct_gput_right:nn #1 #2 %#1 num of parent struct, #2 kid struct
  {
    \@@_seq_gput_right:cx
      { g_@@_struct_kids_#1_seq }
      {
        \pdf_object_ref:n { @@/struct/#2 }
      }
 }

\cs_generate_variant:Nn \@@_struct_kid_struct_gput_right:nn {xx}
%    \end{macrocode}
% \end{macro}
% \begin{macro}
%  {\@@_struct_kid_OBJR_gput_right:nnn,\@@_struct_kid_OBJR_gput_right:xxx}
% At last the command to add an OBJR object. This has to write an object first.
% The first argument is the number of the parent structure, the second the
% (expanded) object reference of the annotation. The last argument is the page
% object reference
%
%    \begin{macrocode}
\cs_new_protected:Npn\@@_struct_kid_OBJR_gput_right:nnn #1 #2 #3 %#1 num of parent struct,
                                                             %#2 obj reference
                                                             %#3 page object reference
  {
    \pdf_object_unnamed_write:nn
      { dict }
      {
        /Type/OBJR/Obj~#2/Pg~#3
      }
    \@@_seq_gput_right:cx
      { g_@@_struct_kids_#1_seq }
      {
        \pdf_object_ref_last:
      }
  }

\cs_generate_variant:Nn\@@_struct_kid_OBJR_gput_right:nnn { xxx }

%    \end{macrocode}
% \end{macro}
% \begin{macro}
%   {\@@_struct_exchange_kid_command:N, \@@_struct_exchange_kid_command:c}
%  In luamode it can happen that a single kid in a structure is split at a page
%  break into two or more mcid. In this case the lua code has to convert
%  put the dictionary of the kid into an array. See issue 13 at tagpdf repo.
%  We exchange the dummy command for the kids to mark this case.
%    \begin{macrocode}
\cs_new_protected:Npn\@@_struct_exchange_kid_command:N #1 %#1 = seq var
  {
    \seq_gpop_left:NN #1 \l_@@_tmpa_tl
    \regex_replace_once:nnN
      { \c{\@@_mc_insert_mcid_kids:n} }
      { \c{\@@_mc_insert_mcid_single_kids:n} }
      \l_@@_tmpa_tl
    \seq_gput_left:NV #1 \l_@@_tmpa_tl
  }

\cs_generate_variant:Nn\@@_struct_exchange_kid_command:N { c }
%    \end{macrocode}
% \end{macro}
% \begin{macro}{ \@@_struct_fill_kid_key:n }
% This command adds the kid info to the K entry. In lua mode the
% content contains commands which are expanded later. The argument is the structure
% number.
%
%    \begin{macrocode}
\cs_new_protected:Npn \@@_struct_fill_kid_key:n #1 %#1 is the struct num
  {
    \bool_if:NF\g_@@_mode_lua_bool
     {
        \seq_clear:N \l_@@_tmpa_seq
        \seq_map_inline:cn { g_@@_struct_kids_#1_seq }
         { \seq_put_right:Nx \l_@@_tmpa_seq { ##1 } }
        %\seq_show:c { g_@@_struct_kids_#1_seq }
        %\seq_show:N \l_@@_tmpa_seq
        \seq_remove_all:Nn \l_@@_tmpa_seq {}
        %\seq_show:N \l_@@_tmpa_seq
        \seq_gset_eq:cN { g_@@_struct_kids_#1_seq } \l_@@_tmpa_seq
     }

    \int_case:nnF
      {
        \seq_count:c
          {
            g_@@_struct_kids_#1_seq
          }
      }
      {
        { 0 }
         { } %no kids, do nothing
        { 1 } % 1 kid, insert
         {
           % in this case we need a special command in
           % luamode to get the array right. See issue #13
           \bool_if:NT\g_@@_mode_lua_bool
             {
               \@@_struct_exchange_kid_command:c
                {g_@@_struct_kids_#1_seq}
             }
           \@@_prop_gput:cnx { g_@@_struct_#1_prop } {K}
             {
               \seq_item:cn
                 {
                   g_@@_struct_kids_#1_seq
                 }
                 {1}
             }
         } %
      }
      { %many kids, use an array
        \@@_prop_gput:cnx { g_@@_struct_#1_prop } {K}
          {
            [
              \seq_use:cn
                {
                  g_@@_struct_kids_#1_seq
                }
                {
                  \c_space_tl
                }
            ]
          }
      }
  }

%    \end{macrocode}
% \end{macro}
%  \subsection{Output of the object}
% \begin{macro}{\@@_struct_get_dict_content:nN}
% This maps the dictionary content of a structure into a tl-var.
% Basically it does what |\pdfdict_use:n| does.
% TODO!! this looks over-complicated. Check if it can be done with pdfdict now.
%    \begin{macrocode}
\cs_new_protected:Npn \@@_struct_get_dict_content:nN #1 #2 %#1: stucture num
  {
    \tl_clear:N #2
    \seq_map_inline:cn
      {
        c_@@_struct_
         \int_compare:nNnTF{#1}={0}{StructTreeRoot}{StructElem}
         _entries_seq
      }
      {
        \tl_put_right:Nx
          #2
          {
             \prop_if_in:cnT
               { g_@@_struct_#1_prop }
               { ##1 }
               {
                 \c_space_tl/##1~
%    \end{macrocode}
% Some keys needs the option to format the key, e.g. add brackets for an
% array
%    \begin{macrocode}
                 \cs_if_exist_use:cTF {@@_struct_format_##1:e}
                   {
                     { \prop_item:cn{ g_@@_struct_#1_prop } { ##1 } }
                   }
                   {
                     \prop_item:cn{ g_@@_struct_#1_prop } { ##1 }
                   }
               }
          }
      }
  }
%    \end{macrocode}
% \end{macro}
% \begin{macro}{\@@_struct_format_Ref:n}
% Ref is an array, we store only the content to be able to extend it
% so the formatting command adds the brackets:
%    \begin{macrocode}
\cs_new:Nn\__tag_struct_format_Ref:n{[#1]}
\cs_generate_variant:Nn\__tag_struct_format_Ref:n{e}
%    \end{macrocode}
% \end{macro}
% \begin{macro}{\@@_struct_write_obj:n}
% This writes out the structure object.
% This is done in the finish code, in the tree module and
% guarded by the tree boolean.
%    \begin{macrocode}
\cs_new_protected:Npn \@@_struct_write_obj:n #1 % #1 is the struct num
  {
    \pdf_object_if_exist:nTF { @@/struct/#1 }
      {
        \@@_struct_fill_kid_key:n { #1 }
        \@@_struct_get_dict_content:nN { #1 } \l_@@_tmpa_tl
        \exp_args:Nx
          \pdf_object_write:nnx
            { @@/struct/#1 }
            {dict}
            {
              \l_@@_tmpa_tl
            }
      }
      {
        \msg_error:nnn { tag } { struct-no-objnum } { #1}
      }
  }
%    \end{macrocode}
% \end{macro}
% \begin{macro}{\@@_struct_insert_annot:nn}
% This is the command to insert an annotation into the structure.
% It can probably be used for xform too.
%
% Annotations used as structure content must
% \begin{enumerate}
% \item add a StructParent integer to their dictionary
% \item push the object reference as OBJR object in the structure
% \item Add a Structparent/obj-nr reference to the parent tree.
% \end{enumerate}
% For a link this looks like this
% \begin{verbatim}
%         \tag_struct_begin:n { tag=Link }
%         \tag_mc_begin:n { tag=Link }
% (1)     \pdfannot_dict_put:nnx
%           { link/URI }
%           { StructParent }
%           { \int_use:N\c@g_@@_parenttree_obj_int }
%    <start link> link text <stop link>
% (2+3)   \@@_struct_insert_annot:nn {obj ref}{parent num}
%         \tag_mc_end:
%         \tag_struct_end:
% \end{verbatim}
%    \begin{macrocode}
\cs_new_protected:Npn \@@_struct_insert_annot:nn #1 #2 %#1 object reference to the annotation/xform
                                                       %#2 structparent number
  {
    \bool_if:NT \g_@@_active_struct_bool
      {
        %get the number of the parent structure:
        \seq_get:NNF
          \g_@@_struct_stack_seq
          \l_@@_struct_stack_parent_tmpa_tl
          {
            \msg_error:nn { tag } { struct-faulty-nesting }
          }
        %put the obj number of the annot in the kid entry, this also creates
        %the OBJR object
        \ref_label:nn {__tag_objr_page_#2 }{ tagabspage }
        \@@_struct_kid_OBJR_gput_right:xxx
          {
            \l_@@_struct_stack_parent_tmpa_tl
          }
          {
            #1 %
          }
          {
            \pdf_pageobject_ref:n { \@@_ref_value:nnn {__tag_objr_page_#2 }{ tagabspage }{1} }
          }
        % add the parent obj number to the parent tree:
        \exp_args:Nnx
        \@@_parenttree_add_objr:nn
          {
            #2
          }
          {
            \pdf_object_ref:e { @@/struct/\l_@@_struct_stack_parent_tmpa_tl }
          }
        % increase the int:
        \stepcounter{ g_@@_parenttree_obj_int }
      }
  }
%    \end{macrocode}
% \end{macro}
%
% \begin{macro}{\@@_get_data_struct_tag:}
% this command allows \cs{tag_get:n} to get the current
% structure tag with the keyword |struct_tag|. We will need to handle nesting
%    \begin{macrocode}
\cs_new:Npn \@@_get_data_struct_tag:
  {
    \exp_args:Ne
    \tl_tail:n
     {
       \prop_item:cn {g_@@_struct_\g_@@_struct_stack_current_tl _prop}{S}
     }
  }
%</package>
%    \end{macrocode}
% \end{macro}
%
% \begin{macro}{\@@_get_data_struct_num:}
% this command allows \cs{tag_get:n} to get the current
% structure number with the keyword |struct_num|. We will need to handle nesting
%    \begin{macrocode}
%<*base>
\cs_new:Npn \@@_get_data_struct_num:
  {
    \g_@@_struct_stack_current_tl
  }
%</base>
%    \end{macrocode}
% \end{macro}
%
% \section{Keys}
% This are the keys for the user commands.
% we store the tag in a variable. But we should be careful, it is only reliable
% at the begin.
% \begin{macro}
%  {
%    label (struct-key),
%    stash (struct-key),
%    parent (struct-key),
%    tag (struct-key),
%    title (struct-key),
%    title-o (struct-key),
%    alt (struct-key),
%    actualtext (struct-key),
%    lang (struct-key),
%    ref (struct-key),
%    E (struct-key)
%  }
%    \begin{macrocode}
%<*package>
\keys_define:nn { @@ / struct }
  {
    label .tl_set:N      = \l_@@_struct_key_label_tl,
    stash .bool_set:N    = \l_@@_struct_elem_stash_bool,
    parent .code:n       =
      {
        \bool_lazy_and:nnTF
          {
            \prop_if_exist_p:c { g_@@_struct_\int_eval:n {#1}_prop }
          }
          {
            \int_compare_p:nNn {#1}<{\c@g_@@_struct_abs_int}
          }
          { \tl_set:Nx \l_@@_struct_stack_parent_tmpa_tl { \int_eval:n {#1} } }
          {
            \msg_warning:nnxx { tag } { struct-unknown }
              { \int_eval:n {#1} }
              { parent~key~ignored }
          }
      },
    parent .default:n    = {-1},
    tag   .code:n        = % S property
      {
        \seq_set_split:Nne \l_@@_tmpa_seq { / } {#1/\prop_item:Ne\g__tag_role_tags_NS_prop{#1}}
        \tl_gset:Nx \g_@@_struct_tag_tl   { \seq_item:Nn\l_@@_tmpa_seq {1} }
        \tl_gset:Nx \g_@@_struct_tag_NS_tl{ \seq_item:Nn\l_@@_tmpa_seq {2} }  
        \@@_check_structure_tag:N \g_@@_struct_tag_tl
      },
    title .code:n        = % T property
      {
        \str_set_convert:Nnnn
          \l_@@_tmpa_str
          { #1 }
          { default }
          { utf16/hex }
        \@@_prop_gput:cnx
          { g_@@_struct_\int_eval:n {\c@g_@@_struct_abs_int}_prop }
          { T }
          { <\l_@@_tmpa_str> }
      },
    title-o .code:n        = % T property
      {
        \str_set_convert:Nonn
          \l_@@_tmpa_str
          { #1 }
          { default }
          { utf16/hex }
        \@@_prop_gput:cnx
          { g_@@_struct_\int_eval:n {\c@g_@@_struct_abs_int}_prop }
          { T }
          { <\l_@@_tmpa_str> }
      },
    alt .code:n      = % Alt property
      {
        \str_set_convert:Noon
          \l_@@_tmpa_str
          { #1 }
          { default }
          { utf16/hex }
        \@@_prop_gput:cnx
          { g_@@_struct_\int_eval:n {\c@g_@@_struct_abs_int}_prop }
          { Alt }
          { <\l_@@_tmpa_str> }
      },
    alttext .meta:n = {alt=#1},
    actualtext .code:n  = % ActualText property
      {
        \str_set_convert:Noon
          \l_@@_tmpa_str
          { #1 }
          { default }
          { utf16/hex }
        \@@_prop_gput:cnx
          { g_@@_struct_\int_eval:n {\c@g_@@_struct_abs_int}_prop }
          { ActualText }
          { <\l_@@_tmpa_str>}
      },
    lang .code:n        = % Lang property
      {
        \@@_prop_gput:cnx
          { g_@@_struct_\int_eval:n {\c@g_@@_struct_abs_int}_prop }
          { Lang }
          { (#1) }
      },
%    \end{macrocode}
% Ref is an array, the brackets are added through the formatting command.
%    \begin{macrocode}
    ref .code:n        = % ref property
      {
        \tl_clear:N\l_@@_tmpa_tl
        \clist_map_inline:on {#1}
          {
            \tl_put_right:Nx \l_@@_tmpa_tl
              {~\ref_value:nn{tagpdfstruct-##1}{tagstructobj} }
          }
        \@@_struct_gput_data_ref:ee { \int_eval:n {\c@g_@@_struct_abs_int} } {\l_@@_tmpa_tl}
      },
    E .code:n        = % E property
      {
        \str_set_convert:Nnon
          \l_@@_tmpa_str
          { #1 }
          { default }
          { utf16/hex }
        \@@_prop_gput:cnx
          { g_@@_struct_\int_eval:n {\c@g_@@_struct_abs_int}_prop }
          { E }
          { <\l_@@_tmpa_str> }
      },
  }
%    \end{macrocode}
% \end{macro}
% \begin{macro}{AF (struct-key),AFinline (struct-key),AFinline-o (struct-key)}
% keys for the AF keys (associated files). They use commands from l3pdffile!
% The stream variants use txt as extension to get the mimetype.
% TODO: check if this should be configurable. For math we will perhaps need another
% extension.
% AF is an array and can be used more than once, so we store it in a tl.
% which is expanded.
% AFinline can be use only once (more quite probably doesn't make sense).
%    \begin{macrocode}
\cs_new_protected:Npn \@@_struct_add_AF:nn #1 #2 % #1 struct num #2 object name
  {
     \tl_if_exist:cTF
       {
         g_@@_struct_#1_AF_tl
       }
       {
         \tl_gput_right:cx
           { g_@@_struct_#1_AF_tl }
           {  ~ \pdf_object_ref:n {#2} }
       }
       {
          \tl_new:c
            { g_@@_struct_#1_AF_tl }
          \tl_gset:cx
            { g_@@_struct_#1_AF_tl }
            { \pdf_object_ref:n {#2} }
       }
  }
\cs_generate_variant:Nn \@@_struct_add_AF:nn {en,ee}
\keys_define:nn { @@ / struct }
 {
    AF .code:n        = % AF property
      {
        \pdf_object_if_exist:nTF {#1}
          {
            \@@_struct_add_AF:en { \int_eval:n {\c@g_@@_struct_abs_int} }{#1}
            \@@_prop_gput:cnx
             { g_@@_struct_\int_eval:n {\c@g_@@_struct_abs_int}_prop }
             { AF }
             {
               [
                 \tl_use:c
                   { g_@@_struct_\int_eval:n {\c@g_@@_struct_abs_int}_AF_tl }
               ]
             }
          }
          {

          }
      },
   ,AFinline .code:n =
     {
       \group_begin:
       \pdf_object_if_exist:eF {@@/fileobj\int_use:N\c@g_@@_struct_abs_int}
         {
           \pdffile_embed_stream:nxx
             {#1}
             {tag-AFfile\int_use:N\c@g_@@_struct_abs_int.txt}
             {@@/fileobj\int_use:N\c@g_@@_struct_abs_int}
           \@@_struct_add_AF:ee
             { \int_eval:n {\c@g_@@_struct_abs_int} }
             { @@/fileobj\int_use:N\c@g_@@_struct_abs_int }
           \@@_prop_gput:cnx
             { g_@@_struct_\int_use:N\c@g_@@_struct_abs_int _prop }
             { AF }
             {
               [
                 \tl_use:c
                  { g_@@_struct_\int_eval:n {\c@g_@@_struct_abs_int}_AF_tl }
               ]
             }
         }
       \group_end:
     }
   ,AFinline-o .code:n =
     {
       \group_begin:
       \pdf_object_if_exist:eF {@@/fileobj\int_use:N\c@g_@@_struct_abs_int}
        {
          \pdffile_embed_stream:oxx
            {#1}
            {tag-AFfile\int_use:N\c@g_@@_struct_abs_int.txt}
            {@@/fileobj\int_use:N\c@g_@@_struct_abs_int}
          \@@_struct_add_AF:ee
             { \int_eval:n {\c@g_@@_struct_abs_int} }
             { @@/fileobj\int_use:N\c@g_@@_struct_abs_int }
           \@@_prop_gput:cnx
             { g_@@_struct_\int_use:N\c@g_@@_struct_abs_int _prop }
             { AF }
             {
               [
                 \tl_use:c
                  { g_@@_struct_\int_eval:n {\c@g_@@_struct_abs_int}_AF_tl }
               ]
             }
        }
       \group_end:
     }
 }
%    \end{macrocode}
% \end{macro}
% \begin{macro}{root-AF (setup-key)}
% The root structure can take AF keys too, so we provide a key for it.
% This key is used with |\tagpdfsetup|, not in a structure!
%    \begin{macrocode}
\keys_define:nn { @@ / setup }
  {
    root-AF .code:n =
     {
        \pdf_object_if_exist:nTF {#1}
          {
            \@@_struct_add_AF:en { 0 }{#1}
            \@@_prop_gput:cnx
             { g_@@_struct_0_prop }
             { AF }
             {
               [
                 \tl_use:c
                   { g_@@_struct_0_AF_tl }
               ]
             }
          }
          {

          }
      },
  }
%</package>
%    \end{macrocode}
% \end{macro}
% \section{User commands}
%
% \begin{macro}{\tag_struct_begin:n,\tag_struct_end:}
%    \begin{macrocode}
%<base>\cs_new_protected:Npn \tag_struct_begin:n #1 {\int_gincr:N \c@g_@@_struct_abs_int}
%<base>\cs_new_protected:Npn \tag_struct_end:{}
%<*package|debug>
%<package>\cs_set_protected:Npn \tag_struct_begin:n #1 %#1 key-val
%<debug>\cs_set_protected:Npn \tag_struct_begin:n #1 %#1 key-val
  {
%<package>\@@_check_if_active_struct:T
%<debug>\@@_check_if_active_struct:TF
      {
        \group_begin:
        \int_gincr:N \c@g_@@_struct_abs_int
        \@@_prop_new:c  { g_@@_struct_\int_eval:n { \c@g_@@_struct_abs_int }_prop }
        \@@_new_output_prop_handler:n {\int_eval:n { \c@g_@@_struct_abs_int }}
        \@@_seq_new:c  { g_@@_struct_kids_\int_eval:n { \c@g_@@_struct_abs_int }_seq}
        \exp_args:Ne
          \pdf_object_new:n
            { @@/struct/\int_eval:n { \c@g_@@_struct_abs_int } }
        \@@_prop_gput:cno
          { g_@@_struct_\int_eval:n { \c@g_@@_struct_abs_int }_prop }
          { Type }
          { /StructElem }
        \tl_set:Nn \l_@@_struct_stack_parent_tmpa_tl {-1}
        \keys_set:nn { @@ / struct} { #1 }
%    \end{macrocode}
%    \begin{macrocode}
        \@@_struct_set_tag_info:nVV 
          { \int_eval:n {\c@g_@@_struct_abs_int} }
           \g_@@_struct_tag_tl               
           \g_@@_struct_tag_NS_tl   
        \@@_check_structure_has_tag:n { \int_eval:n {\c@g_@@_struct_abs_int} }           
        \tl_if_empty:NF
          \l_@@_struct_key_label_tl
          {
            \@@_ref_label:en{tagpdfstruct-\l_@@_struct_key_label_tl}{struct}
          }
%    \end{macrocode}
% The structure number of the parent is either taken from the stack or
% has been set with the parent key.
%    \begin{macrocode}
        \int_compare:nNnT { \l_@@_struct_stack_parent_tmpa_tl } = { -1 }
          {
            \seq_get:NNF
              \g_@@_struct_stack_seq
              \l_@@_struct_stack_parent_tmpa_tl
              {
                \msg_error:nn { tag } { struct-faulty-nesting }
              }
           }
        \seq_gpush:NV \g_@@_struct_stack_seq        \c@g_@@_struct_abs_int
        \seq_gpush:NV \g_@@_struct_tag_stack_seq    \g_@@_struct_tag_tl
        \tl_gset:NV   \g_@@_struct_stack_current_tl \c@g_@@_struct_abs_int
        %\seq_show:N   \g_@@_struct_stack_seq
        \bool_if:NF
          \l_@@_struct_elem_stash_bool
          {
%    \end{macrocode}
%  check if the tag can be used inside the parent. It only makes sense,
%  if the structure is actually used here, so it is guarded by the stash boolean.
%  For now we ignore the namespace! 
%    \begin{macrocode}
           \prop_get:cnNTF 
             { g_@@_struct_ \l_@@_struct_stack_parent_tmpa_tl _ prop }
             {S}
             \l_@@_tmpa_tl
             {
                \cs_set_eq:NN \@@_use_arg:nn\use_ii:nn
                \@@_check_parent_child:eVN 
                  {\l_@@_tmpa_tl} % remove slash
                  \g_@@_struct_tag_tl 
                  \l_@@_parent_child_check_tl                 
                \int_compare:nNnT {\l_@@_parent_child_check_tl}<0 
                 { 
                  %\@@_role_remap:V \l_@@_tmpa_tl 
                 }
             }
             {
               \tl_set:Nn\l_@@_parent_child_check_tl {0}
             }
%    \end{macrocode}
% Set the Parent.
%    \begin{macrocode}
            \@@_prop_gput:cnx
              { g_@@_struct_\int_eval:n {\c@g_@@_struct_abs_int}_prop }
              { P }
              {
                \pdf_object_ref:e { @@/struct/\l_@@_struct_stack_parent_tmpa_tl }
              }
            %record this structure as kid:
            %\tl_show:N \g_@@_struct_stack_current_tl
            %\tl_show:N \l_@@_struct_stack_parent_tmpa_tl
            \@@_struct_kid_struct_gput_right:xx
               { \l_@@_struct_stack_parent_tmpa_tl }
               { \g_@@_struct_stack_current_tl }
            %\prop_show:c { g_@@_struct_\g_@@_struct_stack_current_tl _prop }
            %\seq_show:c {g_@@_struct_kids_\l_@@_struct_stack_parent_tmpa_tl _seq}
          }
        %\prop_show:c { g_@@_struct_\g_@@_struct_stack_current_tl _prop }
        %\seq_show:c {g_@@_struct_kids_\l_@@_struct_stack_parent_tmpa_tl _seq}
%<debug> \@@_debug_struct_begin_insert:n { #1 }
        \group_end:
     }
%<debug>{ \@@_debug_struct_begin_ignore:n { #1 }}
  }
%<package>\cs_set_protected:Nn \tag_struct_end:
%<debug>\cs_set_protected:Nn \tag_struct_end:
  { %take the current structure num from the stack:
    %the objects are written later, lua mode hasn't all needed info yet
    %\seq_show:N \g_@@_struct_stack_seq
%<package>\@@_check_if_active_struct:T
%<debug>\@@_check_if_active_struct:TF
      {
        \seq_gpop:NN   \g_@@_struct_tag_stack_seq \l_@@_tmpa_tl
        \seq_gpop:NNTF \g_@@_struct_stack_seq \l_@@_tmpa_tl
          {
            \@@_check_info_closing_struct:o { \g_@@_struct_stack_current_tl }
          }
          { \@@_check_no_open_struct: }
        % get the previous one, shouldn't be empty as the root should be there
        \seq_get:NNTF \g_@@_struct_stack_seq \l_@@_tmpa_tl
          {
            \tl_gset:NV   \g_@@_struct_stack_current_tl \l_@@_tmpa_tl
          }
          {
            \@@_check_no_open_struct:
          }
       \seq_get:NNT \g_@@_struct_tag_stack_seq \l_@@_tmpa_tl
          {
            \tl_gset:NV \g_@@_struct_tag_tl \l_@@_tmpa_tl
          }
%<debug>\@@_debug_struct_end_insert:
      }
%<debug>{\@@_debug_struct_end_ignore:}
  }
%</package|debug>
%    \end{macrocode}
% \end{macro}
% \begin{macro}{\tag_struct_use:n}
% This command allows to use a stashed structure in another place.
% TODO: decide how it should be guarded. Probably by the struct-check.
%    \begin{macrocode}
%<base>\cs_new_protected:Npn \tag_struct_use:n #1 {}
%<*package>
\cs_set_protected:Npn \tag_struct_use:n #1 %#1 is the label
  {
    \@@_check_if_active_struct:T
      {
        \prop_if_exist:cTF
          { g_@@_struct_\@@_ref_value:enn{tagpdfstruct-#1}{tagstruct}{unknown}_prop } %
          {
            \@@_check_struct_used:n {#1}
            %add the label structure as kid to the current structure (can be the root)
            \@@_struct_kid_struct_gput_right:xx
              { \g_@@_struct_stack_current_tl }
              { \@@_ref_value:enn{tagpdfstruct-#1}{tagstruct}{0} }
            %add the current structure to the labeled one as parents
            \@@_prop_gput:cnx
              { g_@@_struct_\@@_ref_value:enn{tagpdfstruct-#1}{tagstruct}{0}_prop }
              { P }
              {
                \pdf_object_ref:e { @@/struct/\g_@@_struct_stack_current_tl }
              }
          }
          {
            \msg_warning:nnn{ tag }{struct-label-unknown}{#1}
          }
      }
  }
%</package>
%    \end{macrocode}
% \end{macro}
%
% \begin{macro}[EXP]{\tag_struct_object_ref:n}
% This is a command that allows to reference a structure. The argument is the
% number which can be get for the current structure with |\tag_get:n{struct_num}|
% TODO check if it should be in base too.
%    \begin{macrocode}
%<*package>
\cs_new:Npn \tag_struct_object_ref:n #1
 {
   \pdf_object_ref:n {@@/struct/#1}
 }
\cs_generate_variant:Nn \tag_struct_object_ref:n {e}
%    \end{macrocode}
%
% \end{macro}
%
% \begin{macro}{\tag_struct_gput:nnn}
% This is a command that allows to update the data of a structure.
% The first argument is the
% number of the structure, the second a keyword referring to a function,
% the third the value. Currently the only keyword is \texttt{ref}
%    \begin{macrocode}
\cs_new_protected:Npn \tag_struct_gput:nnn #1 #2 #3
 {
   \cs_if_exist_use:cF {@@_struct_gput_data_#2:nn}
    { %warning??
      \use_none:nn
    }
    {#1}{#3}
 }
\cs_generate_variant:Nn \tag_struct_gput:nnn {ene,nne}
%    \end{macrocode}
% \end{macro}
%
% \begin{macro}{\@@_struct_gput_data_ref:nn}
%    \begin{macrocode}
\cs_new_protected:Npn \@@_struct_gput_data_ref:nn #1 #2
   % #1 receiving struct num, #2 list of object ref
   {
     \prop_get:cnN
        { g_@@_struct_#1_prop }
        {Ref}
        \l_@@_tmpb_tl
     \@@_prop_gput:cnx
        { g_@@_struct_#1_prop }
        { Ref }
        { \quark_if_no_value:NF\l_@@_tmpb_tl { \l_@@_tmpb_tl\c_space_tl }#2 }
    }
\cs_generate_variant:Nn \@@_struct_gput_data_ref:nn {ee}
%    \end{macrocode}
% \end{macro}

% \begin{macro}
%   {
%     \tag_struct_insert_annot:nn,
%     \tag_struct_insert_annot:xx
%   }
% \begin{macro}[EXP]
%   {
%     \tag_struct_parent_int:
%   }
% This are the user command to insert annotations. They must be used
% together to get the numbers right. They use a counter to the
% |StructParent| and \cs{tag_struct_insert_annot:nn}  increases the
% counter given back by \cs{tag_struct_parent_int:}.
%
% It must be used together with |\tag_struct_parent_int:| to insert an
% annotation.
% TODO: decide how it should be guarded if tagging is deactivated.
%    \begin{macrocode}
\cs_new_protected:Npn \tag_struct_insert_annot:nn #1 #2 %#1 should be an object reference
                                                        %#2 struct parent num
  {
    \@@_check_if_active_struct:T
      {
        \@@_struct_insert_annot:nn {#1}{#2}
      }
  }

\cs_generate_variant:Nn \tag_struct_insert_annot:nn {xx}
\cs_new:Npn \tag_struct_parent_int: {\int_use:c { c@g_@@_parenttree_obj_int }}

%</package>

%    \end{macrocode}
% \end{macro}
% \end{macro}
% \section{Attributes and attribute classes}
%    \begin{macrocode}
%<*header>
\ProvidesExplPackage {tagpdf-attr-code} {2022-08-24} {0.97}
  {part of tagpdf - code related to attributes and attribute classes}
%</header>
%    \end{macrocode}
% \subsection{Variables}
%  \begin{variable}
%   {
%     ,\g_@@_attr_entries_prop
%     ,\g_@@_attr_class_used_seq
%     ,\g_@@_attr_objref_prop
%     ,\l_@@_attr_value_tl
%   }
% |\g_@@_attr_entries_prop| will store attribute names and their dictionary content.\\
% |\g_@@_attr_class_used_seq| will hold the attributes which have been used as
% class name.
% |\l_@@_attr_value_tl| is used to build the attribute array or key.
% Everytime an attribute is used for the first time, and object is created
% with its content, the name-object reference relation is stored in
% |\g_@@_attr_objref_prop|
%    \begin{macrocode}
%<*package>
\prop_new:N \g_@@_attr_entries_prop
\seq_new:N  \g_@@_attr_class_used_seq
\tl_new:N   \l_@@_attr_value_tl
\prop_new:N \g_@@_attr_objref_prop %will contain obj num of used attributes
%    \end{macrocode}
% \end{variable}
% \subsection{Commands and keys}
% \begin{macro}{\@@_attr_new_entry:nn,newattribute (setup-key)}
% This allows to define attributes. Defined attributes
% are stored in a global property. |newattribute| expects
% two brace group, the name and the content. The content typically
% needs an |/O| key for the owner. An example look like
% this.
% \begin{verbatim}
%  \tagpdfsetup
%   {
%    newattribute =
%     {TH-col}{/O /Table /Scope /Column},
%    newattribute =
%     {TH-row}{/O /Table /Scope /Row},
%    }
% \end{verbatim}
%    \begin{macrocode}
\cs_new_protected:Npn \@@_attr_new_entry:nn #1 #2 %#1:name, #2: content
  {
    \prop_gput:Nen \g_@@_attr_entries_prop
      {\pdf_name_from_unicode_e:n{#1}}{#2}
  }

\keys_define:nn { @@ / setup }
  {
    newattribute .code:n =
      {
        \@@_attr_new_entry:nn #1
      }
  }
%    \end{macrocode}
% \end{macro}
% \begin{macro}{attribute-class (struct-key)}
% attribute-class has to store the used attribute names so that
% they can be added to the ClassMap later.
%    \begin{macrocode}
\keys_define:nn { @@ / struct }
  {
    attribute-class .code:n =
     {
       \clist_set:No \l_@@_tmpa_clist { #1 }
       \seq_set_from_clist:NN \l_@@_tmpb_seq \l_@@_tmpa_clist
%    \end{macrocode}
%    we convert the names into pdf names with slash
%    \begin{macrocode}
       \seq_set_map_x:NNn \l_@@_tmpa_seq \l_@@_tmpb_seq
         {
           \pdf_name_from_unicode_e:n {##1}
         }
       \seq_map_inline:Nn \l_@@_tmpa_seq
         {
           \prop_if_in:NnF \g_@@_attr_entries_prop {##1}
             {
               \msg_error:nnn { tag } { attr-unknown } { ##1 }
             }
           \seq_gput_left:Nn\g_@@_attr_class_used_seq { ##1}
         }
       \tl_set:Nx \l_@@_tmpa_tl
         {
           \int_compare:nT { \seq_count:N \l_@@_tmpa_seq > 1 }{[}
           \seq_use:Nn \l_@@_tmpa_seq  { \c_space_tl  }
           \int_compare:nT { \seq_count:N \l_@@_tmpa_seq > 1 }{]}
         }
       \int_compare:nT { \seq_count:N \l_@@_tmpa_seq > 0 }
         {
           \@@_prop_gput:cnx
             { g_@@_struct_\int_eval:n {\c@g_@@_struct_abs_int}_prop }
             { C }
             { \l_@@_tmpa_tl }
          %\prop_show:c  { g_@@_struct_\int_eval:n {\c@g_@@_struct_abs_int}_prop }
         }
     }
  }
%    \end{macrocode}
% \end{macro}
% \begin{macro}{attribute (struct-key)}
%    \begin{macrocode}
\keys_define:nn { @@ / struct }
  {
    attribute .code:n  = % A property (attribute, value currently a dictionary)
      {
        \clist_set:No          \l_@@_tmpa_clist { #1 }
        \seq_set_from_clist:NN \l_@@_tmpb_seq \l_@@_tmpa_clist
%    \end{macrocode}
%    we convert the names into pdf names with slash
%    \begin{macrocode}
       \seq_set_map_x:NNn \l_@@_tmpa_seq \l_@@_tmpb_seq
         {
           \pdf_name_from_unicode_e:n {##1}
         }
        \tl_set:Nx \l_@@_attr_value_tl
          {
            \int_compare:nT { \seq_count:N \l_@@_tmpa_seq > 1 }{[}%]
          }
        \seq_map_inline:Nn \l_@@_tmpa_seq
          {
            \prop_if_in:NnF \g_@@_attr_entries_prop {##1}
              {
                \msg_error:nnn { tag } { attr-unknown } { ##1 }
              }
            \prop_if_in:NnF \g_@@_attr_objref_prop {##1}
              {%\prop_show:N \g_@@_attr_entries_prop
                \pdf_object_unnamed_write:nx
                  { dict }
                  {
                    \prop_item:Nn\g_@@_attr_entries_prop {##1}
                  }
                \prop_gput:Nnx \g_@@_attr_objref_prop {##1} {\pdf_object_ref_last:}
              }
            \tl_put_right:Nx \l_@@_attr_value_tl
              {
                \c_space_tl
                \prop_item:Nn \g_@@_attr_objref_prop {##1}
              }
 %     \tl_show:N \l_@@_attr_value_tl
          }
        \tl_put_right:Nx \l_@@_attr_value_tl
          { %[
            \int_compare:nT { \seq_count:N \l_@@_tmpa_seq > 1 }{]}%
          }
 %     \tl_show:N \l_@@_attr_value_tl
        \@@_prop_gput:cnx
          { g_@@_struct_\int_eval:n {\c@g_@@_struct_abs_int}_prop }
          { A }
          { \l_@@_attr_value_tl }
    },
  }
%</package>
%    \end{macrocode}
% \end{macro}
% \end{implementation}

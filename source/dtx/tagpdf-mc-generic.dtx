% \iffalse meta-comment
%
%% File: tagpdf-mc.dtx
%
% Copyright (C) 2019-2021 Ulrike Fischer
%
% It may be distributed and/or modified under the conditions of the
% LaTeX Project Public License (LPPL), either version 1.3c of this
% license or (at your option) any later version.  The latest version
% of this license is in the file
%
%    https://www.latex-project.org/lppl.txt
%
% This file is part of the "tagpdf bundle" (The Work in LPPL)
% and all files in that bundle must be distributed together.
%
% -----------------------------------------------------------------------
%
% The development version of the bundle can be found at
%
%    https://github.com/u-fischer/tagpdf
%
% for those people who are interested.
%<*driver>
\RequirePackage{pdfmanagement-testphase}
\DeclareDocumentMetadata{}
\makeatletter
\declare@file@substitution{doc.sty}{doc-v3beta.sty}
\makeatother
\documentclass{l3doc}
\usepackage{array,booktabs,caption}
\hypersetup{pdfauthor=Ulrike Fischer,
 pdftitle=tagpdf-mc module (tagpdf)}
\begin{document}
  \DocInput{\jobname.dtx}
\end{document}
%</driver>
% \fi
% \title{^^A
%   The \pkg{tagpdf-mc-generic} module\\ Code related to Marked Content (mc-chunks), generic mode  ^^A
%   \\ Part of the tagpdf package
% }
%
% \author{^^A
%  Ulrike Fischer\thanks
%    {^^A
%      E-mail:
%        \href{mailto:fischer@troubleshooting-tex.de}
%          {fischer@troubleshooting-tex.de}^^A
%    }^^A
% }
%
% \date{Version 0.91, released 2021-07-03}
% \maketitle
% \begin{documentation}
% \begin{function}{\tag_mc_begin_single:nN}
%  \begin{syntax}
%   \cs{tag_mc_begin_single:nN} \Arg{tag}\meta{tl-var}\\
%   \cs{tag_mc_end_single:}
%  \end{syntax}
% These two functions allow to inject an end and begin MC during the output
% routine to close and reopen an mc from a page break.
% \end{function}
% \begin{function}{\tag_mc_store:nn}
%  \begin{syntax}
%  \cs{tag_mc_store:nn}\Arg{mc-num}\Arg{struct-num}
%  \end{syntax}
%  This inserts the mc-chunk \meta{mc-num} into the structure {struct-num}.
%  The structure must already exist. The mc-chunk is added at the end.
%  This is a preliminary minimal function and will change!
%  TODO: this function must be expanded to allow to insert the chunk
%  also in the middle, and perhaps also to insert by label.
% \end{function}
% \begin{function}{\tag_mc_topmarks,\tag_mc_firstmarks:,\tag_mc_botmarks:}
%  These functions retrieve the marks set by the |\tag_mc|-commands.
% \end{function}
% \end{documentation}
% \begin{implementation}
% \section{Marked content code -- generic mode}
%    \begin{macrocode}
%<@@=tag>
%<*generic>
\ProvidesExplPackage {tagpdf-mc-code-generic} {2021-07-03} {0.91}
 {part of tagpdf - code related to marking chunks - generic mode}
%</generic>
%    \end{macrocode}
%
%\subsection{Variables}
%
% \begin{variable}{\g_@@_MCID_byabspage_prop}
% This property will hold the current maximum on a page
% it will contain key-value of type \meta{abspagenum}=\meta{max mcid}
%
%    \begin{macrocode}
%<*generic>
\@@_prop_new:N \g_@@_MCID_byabspage_prop
%    \end{macrocode}
% \end{variable}

% \begin{variable}{\l_@@_mc_ref_abspage_tl}
% We need a ref-label system to ensure that the MCID cnt
% restarts at 0 on a new page
% This will be used to store the tagabspage attribute retrieved from
% a label.
%    \begin{macrocode}
\tl_new:N \l_@@_mc_ref_abspage_tl
%    \end{macrocode}
% \end{variable}

% \begin{variable}{\l_@@_mc_tmpa_tl}
% temporary variable
%    \begin{macrocode}
\tl_new:N \l_@@_mc_tmpa_tl
%    \end{macrocode}
% \end{variable}

% \begin{variable}{\g_@@_mc_marks}
% a marks register
%    \begin{macrocode}
\newmarks \g_@@_mc_marks
%    \end{macrocode}
% \end{variable}

% \subsection{Functions}
%
% \begin{macro}{\@@_mc_begin_mark:nn,\@@_mc_end_mark:}
% Generic mode need to set marks for the page break handling
%    \begin{macrocode}
\cs_new_protected:Npn \@@_mc_begin_mark:nn #1 #2 %#1 tag, #2 label
  {
    \marks\g_@@_mc_marks
      {
        \int_use:N\c@g_@@_MCID_abs_int, %mc-num
        \g_@@_struct_stack_current_tl,  %structure num
        begin,
        #1, %tag
        \bool_if:NTF \l_@@_mc_key_stash_bool{true}{false}, % stash info
        #2, %label
      }
  }

\cs_new_protected:Npn \@@_mc_end_mark:
  {
    \marks\g_@@_mc_marks
      {
        \int_use:N\c@g_@@_MCID_abs_int, %mc-num
        \g_@@_struct_stack_current_tl,  %structure num
        end
      }
  }

%    \end{macrocode}
% \end{macro}
% \begin{macro}{\tag_mc_botmarks:,\tag_mc_firstmarks:,\tag_mc_topmarks:}
%    \begin{macrocode}
\cs_new:Npn\tag_mc_topmarks: {\topmarks\g__tag_mc_marks}
\cs_new:Npn\tag_mc_botmarks: {\botmarks\g__tag_mc_marks}
\cs_new:Npn\tag_mc_firstmarks: {\firstmarks\g__tag_mc_marks}
%    \end{macrocode}
% \end{macro}
% \begin{macro}{\tag_mc_begin_single:n,\tag_mc_end_single:n}
% We also need two functions to inject an end and begin MC during the output
% routine to handle the page break.
% They shouldn't set the booleans and do tests.
% The begin command insert the literal and creates
% the needed PDF objects, increases the absolute counter, and return its values.
% The end command inserts only the literal.
% TODO: should they get public names?
%    \begin{macrocode}
\cs_new_protected:Npn \tag_mc_begin_single:nN #1 #2 %#1 tag, #2 return value
  {
    \__tag_mc_bdc_mcid:n{#1} %
    \tl_set:Nx #2 {\int_eval:n{\c@g__tag_MCID_abs_int}}% store number
  }

\cs_new_protected:Npn \tag_mc_end_single:
 {
   \__tag_mc_emc:
 }
%    \end{macrocode}
% \end{macro}
% \begin{macro}{\tag_mc_store:nn}
%  This inserts the mc-chunk \meta{mc-num} into the structure {struct-num}.
%  The structure must already exist. The mc-chunk is added at the end.
%  This is a preliminary minimal function and will change!
%    \begin{macrocode}
\cs_new_protected:Npn \tag_mc_store:nn #1 #2 %#1 mc-num #2 structure-num
  {
    \__tag_struct_kid_mc_gput_right:nx
      {#2}
      {#1}
    \prop_gput:Nxx \g__tag_mc_parenttree_prop
      {#1}
      {#2}
  }
\cs_generate_variant:Nn \tag_mc_store:nn {xx}
%    \end{macrocode}
% \end{macro}
% \begin{macro}[pTF]{\@@_mc_if_in:,\tag_mc_if_in:}
% This is a test if a mc is open or not. It depends simply on a global boolean:
% mc-chunks are added linearly so nesting should not be relevant.
%    \begin{macrocode}
\prg_new_conditional:Nnn \@@_mc_if_in: {p,T,F,TF}
  {
    \bool_if:NTF \g_@@_in_mc_bool
      { \prg_return_true:  }
      { \prg_return_false: }
  }

\prg_new_eq_conditional:NNn \tag_mc_if_in: \@@_mc_if_in: {p,T,F,TF}
%    \end{macrocode}
% \end{macro}
%
% \begin{macro}{ \@@_mc_bmc:n,\@@_mc_emc:,\@@_mc_bdc:nn,\@@_mc_bdc:nx}
% These are the low-level commands. There are now equal to the
% pdfmanagement commands generic mode, but we use an indirection
% in case luamode need something else.
% change 04.08.2018: the commands do not check the validity of the arguments or try
% to escape them, this should be done before using them.
%    \begin{macrocode}
% #1 tag, #2 properties
\cs_set_eq:NN \@@_mc_bmc:n  \pdf_bmc:n
\cs_set_eq:NN \@@_mc_emc:   \pdf_emc:
\cs_set_eq:NN \@@_mc_bdc:nn \pdf_bdc:nn
\cs_generate_variant:Nn \@@_mc_bdc:nn {nx}
%    \end{macrocode}
% \end{macro}
%
% \begin{macro}
%   {
%     \@@_mc_bdc_mcid:nn,\@@_mc_bdc_mcid:n,
%     \@@_mc_handle_mcid:nn,\@@_mc_handle_mcid:VV
%   }
%
% This create a BDC mark with an |/MCID| key. Most of the work here is to get
% the current number value for the MCID: they must be numbered by page
% starting with 0 and then successively.
% The first argument is the tag, e.g. |P| or |Span|, the second is used to pass
% more properties.
% We also define a wrapper around the low-level command as luamode will need
% something different.
%    \begin{macrocode}
\cs_new_protected:Npn \@@_mc_bdc_mcid:nn #1 #2
  {
    \int_gincr:N \c@g_@@_MCID_abs_int
    \tl_set:Nx \l_@@_mc_ref_abspage_tl
      {
        \@@_ref_value:enn %3 args
          {
            mcid-\int_use:N \c@g_@@_MCID_abs_int
          }
          { tagabspage }
          {-1}
      }
    \prop_get:NoNTF
      \g_@@_MCID_byabspage_prop
      {
        \l_@@_mc_ref_abspage_tl
      }
      \l_@@_mc_tmpa_tl
      {
        %key already present, use value for MCID and add 1 for the next
        \int_gset:Nn \g_@@_MCID_tmp_bypage_int { \l_@@_mc_tmpa_tl }
        \@@_prop_gput:Nxx
          \g_@@_MCID_byabspage_prop
          { \l_@@_mc_ref_abspage_tl }
          { \int_eval:n {\l_@@_mc_tmpa_tl +1} }
      }
      {
        %key not present, set MCID to 0 and insert 1
        \int_gzero:N \g_@@_MCID_tmp_bypage_int
        \@@_prop_gput:Nxx
          \g_@@_MCID_byabspage_prop
          { \l_@@_mc_ref_abspage_tl }
          {1}
      }
    \@@_ref_label:en
      {
        mcid-\int_use:N \c@g_@@_MCID_abs_int
      }
      { mc }
     \@@_mc_bdc:nx
       {#1}
       { /MCID~\int_eval:n { \g_@@_MCID_tmp_bypage_int }~ \exp_not:n { #2 } }
 }
\cs_new_protected:Npn \@@_mc_bdc_mcid:n #1
  {
    \@@_mc_bdc_mcid:nn {#1} {}
  }

\cs_new_protected:Npn \@@_mc_handle_mcid:nn #1 #2 %#1 tag, #2 properties
  {
    \@@_mc_bdc_mcid:nn {#1} {#2}
  }

\cs_generate_variant:Nn \@@_mc_handle_mcid:nn {VV}
%    \end{macrocode}
% \end{macro}

% \begin{macro}{\@@_mc_handle_stash:n,\@@_mc_handle_stash:x}
% This is the handler which puts a mc into the
% the current structure. The argument is the number of the mc.
% Beside storing the mc into the structure, it also has to record the
% structure for the parent tree.
% The name is a bit confusing, it does \emph{not} handle mc with the stash key
% \ldots.
% TODO: why does luamode use it for begin + use, but generic mode only for begin?
%    \begin{macrocode}
\cs_new_protected:Npn \@@_mc_handle_stash:n #1 %1 mcidnum
  {
    \@@_check_mc_used:n {#1}
    \@@_struct_kid_mc_gput_right:nn
      { \g_@@_struct_stack_current_tl }
      {#1}
   \prop_gput:Nxx \g_@@_mc_parenttree_prop
     {#1}
     { \g_@@_struct_stack_current_tl }
  }
\cs_generate_variant:Nn \@@_mc_handle_stash:n { x }
%    \end{macrocode}
% \end{macro}

% \begin{macro}
%  {
%    \@@_mc_bmc_artifact:,
%    \@@_mc_bmc_artifact:n,
%    \@@_mc_handle_artifact:N
%   }
% Two commands to create artifacts, one without type, and one with.
% We define also a wrapper handler as luamode will need a different definition.
% TODO: perhaps later: more properties for artifacts
%    \begin{macrocode}
\cs_new_protected:Npn  \@@_mc_bmc_artifact:
  {
    \@@_mc_bmc:n {Artifact}
  }
\cs_new_protected:Npn \@@_mc_bmc_artifact:n #1
  {
    \@@_mc_bdc:nn {Artifact}{/Type/#1}
  }
\cs_new_protected:Npn \@@_mc_handle_artifact:N #1
   % #1 is a var containing the artifact type
  {
    \tl_if_empty:NTF #1
      { \@@_mc_bmc_artifact: }
      { \exp_args:NV\@@_mc_bmc_artifact:n #1 }
  }
%    \end{macrocode}
% \end{macro}

% \begin{macro}{ \@@_get_data_mc_tag: }
% This allows to retrieve the active mc-tag.
% It is use by the get command.
%    \begin{macrocode}
\cs_new:Nn \@@_get_data_mc_tag: { \g_@@_mc_key_tag_tl }
%    \end{macrocode}
% \end{macro}
%
% \begin{macro}{\tag_mc_begin:n,\tag_mc_end:}
% These are the core public commands to open and close an mc.
% They don't need to be in the same group or grouping level,
% but the code expect that they are issued linearly. The tag and
% the state is passed to the end command through a global var and
% a global boolean.
%    \begin{macrocode}
\cs_new_protected:Npn \tag_mc_begin:n #1 %#1 keyval
  {
    \@@_check_if_active_mc:T
      {
        \group_begin: %hm
        \@@_check_mc_if_nested:
        \bool_gset_true:N \g_@@_in_mc_bool
        \keys_set:nn { @@ / mc } {#1}
        \bool_if:NTF \l_@@_mc_artifact_bool
          { %handle artifact
            \@@_mc_handle_artifact:N \l_@@_mc_artifact_type_tl
          }
          { %handle mcid type
            \@@_check_mc_tag:N  \l_@@_mc_key_tag_tl
            \@@_mc_handle_mcid:VV
               \l_@@_mc_key_tag_tl
               \l_@@_mc_key_properties_tl
            \@@_mc_begin_mark:nn {\l_@@_mc_key_tag_tl}{\l_@@_mc_key_label_tl}
            \tl_if_empty:NF {\l_@@_mc_key_label_tl}
              {
                \exp_args:NV
                \@@_mc_handle_mc_label:n \l_@@_mc_key_label_tl
              }
            \bool_if:NF \l_@@_mc_key_stash_bool
              {
                \@@_mc_handle_stash:x { \int_use:N \c@g_@@_MCID_abs_int }
              }
          }
        \group_end:
      }
  }
\cs_new_protected:Nn \tag_mc_end:
  {
    \@@_check_if_active_mc:T
      {
        \@@_check_mc_if_open:
        \bool_gset_false:N \g_@@_in_mc_bool
        \tl_gset:Nn  \g_@@_mc_key_tag_tl { }
        \@@_mc_emc:
        \@@_mc_end_mark:
      }
  }
%    \end{macrocode}
% \end{macro}

%
% \subsection{Keys}
% Definitions are different in luamode.
% |tag| and |raw| are expanded as |\lua_now:e| in lua does it too and
% we assume that their values are safe.
% \begin{macro}
%  {
%   tag,raw,
%   alttext,alttext-o,
%   actualtext,actualtext-o,
%   label,artifact
%  }
%    \begin{macrocode}
\keys_define:nn { @@ / mc }
  {
    tag .code:n = % the name (H,P,Span) etc
      {
        \tl_set:Nx   \l_@@_mc_key_tag_tl { #1 }
        \tl_gset:Nx  \g_@@_mc_key_tag_tl { #1 }
      },
    raw  .code:n =
      {
        \tl_put_right:Nx \l_@@_mc_key_properties_tl { #1 }
      },
    alttext .code:n = % Alt property
      {
        \str_set_convert:Nnon
          \l_@@_tmpa_str
          { #1 }
          { default }
          { utf16/hex }
        \tl_put_right:Nn \l_@@_mc_key_properties_tl { /Alt~< }
        \tl_put_right:No \l_@@_mc_key_properties_tl { \l_@@_tmpa_str>~ }
      },
    alttext-o .code:n      = % Alt property
      {
        \str_set_convert:Noon
          \l_@@_tmpa_str
          { #1 }
          { default }
          { utf16/hex }
        \tl_put_right:Nn \l_@@_mc_key_properties_tl { /Alt~< }
        \tl_put_right:No \l_@@_mc_key_properties_tl { \l_@@_tmpa_str>~ }
      },
    actualtext .code:n      = % ActualText property
      {
        \str_set_convert:Nnon
          \l_@@_tmpa_str
          { #1 }
          { default }
          { utf16/hex }
        \tl_put_right:Nn \l_@@_mc_key_properties_tl { /ActualText~< }
        \tl_put_right:No \l_@@_mc_key_properties_tl { \l_@@_tmpa_str>~ }
      },
    actualtext-o .code:n      = % ActualText property
      {
        \str_set_convert:Noon
          \l_@@_tmpa_str
          { #1 }
          { default }
          { utf16/hex }
        \tl_put_right:Nn \l_@@_mc_key_properties_tl { /ActualText~< }
        \tl_put_right:No \l_@@_mc_key_properties_tl { \l_@@_tmpa_str>~ }
      },
    label .tl_set:N        = \l_@@_mc_key_label_tl,
    artifact .code:n       =
      {
        \exp_args:Nnx
          \keys_set:nn
            { @@ / mc }
            { __artifact-bool, __artifact-type=#1 }
      },
    artifact .default:n    = {notype}
  }
%</generic>
%    \end{macrocode}
% \end{macro}
% \end{implementation}
% \PrintIndex

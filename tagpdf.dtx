% \iffalse meta-comment
%
%% File: tagpdf.dtx
%
% Copyright (C) 2019-2024 Ulrike Fischer
%
% It may be distributed and/or modified under the conditions of the
% LaTeX Project Public License (LPPL), either version 1.3c of this
% license or (at your option) any later version.  The latest version
% of this license is in the file
%
%    https://www.latex-project.org/lppl.txt
%
% This file is part of the "tagpdf bundle" (The Work in LPPL)
% and all files in that bundle must be distributed together.
%
% -----------------------------------------------------------------------
%
% The development version of the bundle can be found at
%
%    https://github.com/latex3/tagpdf
%
% for those people who are interested.
%
% \fi
% \begin{documentation}
% \begin{function}{
%  \tag_stop:, \tag_start:,
%  \tagstop, \tagstart }
% We need commands to stop tagging in some places.
% They switches three local booleans and also stop the counting
% of paragraphs. If they are nested an inner \cs{tag_start:} will not
% restart tagging.
% \end{function}
%
% \begin{function}{\tag_stop:n, \tag_start:n }%
% \begin{syntax}
% \cs{tag_stop:n}\Arg{label}
% \cs{tag_start:n}\Arg{label}\\
% \end{syntax}
% The commands with argument allow to give a label. This is only used
% in debugging messages to allow to follow the nesting.
% \end{function}
%
% \begin{function}{activate-space (setup-key)}
%  |activate-space| activates the additional parsing needed for
%  interword spaces. It is not documented, the parsing is currently
%  implicitly activated by the known key |interwordspace|, as the code will
%  perhaps move to some other place, now that it is better separated.
% \end{function}
%
% \begin{function}{activate-mc (setup-key),
%    activate-tree (setup-key),
%    activate-struct (setup-key),
%    activate-all (setup-key)}
% Keys to activate the various tagging steps
% \end{function}
%\begin{function}{no-struct-dest (setup-key)}
% The key allows to suppress the creation of structure destinations
% \end{function}
%
% \begin{function}{log (setup-key)}
% The |log| takes currently the values |none|, |v|, |vv|, |vvv|, |all|.
% More details are in \texttt{tagpdf-checks}.
% \end{function}
%
% \begin{function}{tagunmarked (setup-key)}
% This key allows to set if (in luamode) unmarked text should be
% marked up as artifact. The initial value is true.
% \end{function}
%
% \begin{function}{tabsorder (setup-key)}
% This sets the tabsorder on a page. The values are |row|, |column|, |structure| (default)
% or |none|. Currently this is set more or less globally. More finer control can be
% added if needed.
% \end{function}
%
% \begin{function}{tagstruct,tagstructobj,tagabspage,tagmcabs,tagmcid}
% These are attributes used by the label/ref system.
% \end{function}
% \end{documentation}
% \begin{implementation}
% \section{Initialization and test if pdfmanagement is active.}
%    \begin{macrocode}
%<@@=tag>
%<*package>
\ProvidesExplPackage {tagpdf} {2024-01-26} {0.98t}
  { A package to experiment with pdf tagging }

\bool_if:nF
  {
    \bool_lazy_and_p:nn
      {\cs_if_exist_p:N \pdfmanagement_if_active_p:}
      { \pdfmanagement_if_active_p: }
  }
  {  %error for now, perhaps warning later.
    \PackageError{tagpdf}
     {
       PDF~resource~management~is~no~active!\MessageBreak
       tagpdf~will~no~work.
     }
     {
       Activate~it~with \MessageBreak
       \string\RequirePackage{pdfmanagement-testphase}\MessageBreak
       \string\DocumentMetadata{<options>}\MessageBreak
       before~\string\documentclass
     }
  }
%</package>
%    \end{macrocode}
%<*debug>
%    \begin{macrocode}
\ProvidesExplPackage {tagpdf-debug} {2024-01-26} {0.98t}
  { debug code for tagpdf }
\@ifpackageloaded{tagpdf}{}{\PackageWarning{tagpdf-debug}{tagpdf~not~loaded,~quitting}\endinput}
%    \end{macrocode}
%</debug>
% We map the internal module name \enquote{tag} to \enquote{tagpdf} in messages.
%    \begin{macrocode}
%<*package>
\prop_gput:Nnn \g_msg_module_name_prop { tag }{ tagpdf }
%</package>
%    \end{macrocode}
% Debug mode has its special mapping:
%    \begin{macrocode}
%<*debug>
\prop_gput:Nnn \g_msg_module_type_prop { tag / debug} {}
\prop_gput:Nnn \g_msg_module_name_prop { tag / debug }{tagpdf~DEBUG}
%</debug>
%    \end{macrocode}
%\section{base package}
% To avoid to have to test everywhere if tagpdf has been loaded and is active,
% we define a base package with dummy functions
%    \begin{macrocode}
%<*base>
\ProvidesExplPackage {tagpdf-base} {2024-01-26} {0.98t}
  {part of tagpdf - provide base, no-op versions of the user commands }
%</base>
%    \end{macrocode} 
%\section{Package options}
% There are only two documented options to switch for luatex between generic and luamode,
% TODO try to get rid of them.
% The option \texttt{disabledelayedshipout} is only temporary to be able to debug 
% problem with the new shipout keyword if needed.
%    \begin{macrocode}
%<*package>
\bool_new:N\g_@@_mode_lua_bool  
\bool_new:N\g_@@_delayed_shipout_bool
\bool_lazy_and:nnT 
  { \bool_if_exist_p:N \l__pdfmanagement_delayed_shipout_bool }
  { \l__pdfmanagement_delayed_shipout_bool }
  {
    \bool_gset_true:N\g_@@_delayed_shipout_bool
  }
\DeclareOption {luamode}  { \sys_if_engine_luatex:T { \bool_gset_true:N \g_@@_mode_lua_bool } }
\DeclareOption {genericmode}{ \bool_gset_false:N\g_@@_mode_lua_bool }
\DeclareOption {disabledelayedshipout}{ \bool_gset_false:N\g_@@_delayed_shipout_bool }
\ExecuteOptions{luamode}
\ProcessOptions
%    \end{macrocode}
% \section{Packages}
% To be on the safe side for now, load also the base definitions
%    \begin{macrocode}
\RequirePackage{tagpdf-base}
%</package>
%    \end{macrocode}
% The no-op version should behave a near enough to the real code as
% possible, so we define a command which a special in the relevant backends:
%    \begin{macrocode}
%<*base>
\AddToHook{begindocument}
 {
  \str_case:VnF \c_sys_backend_str
   {
    { luatex  } { \cs_new_protected:Npn \@@_whatsits: {} }
    { dvisvgm } { \cs_new_protected:Npn \@@_whatsits: {} }
   }
   {
     \cs_new_protected:Npn \@@_whatsits: {\tex_special:D {} }
   }
 }
%</base>
%    \end{macrocode}

% \subsection{a LastPage label}
% See also issue \#2 in Accessible-xref
% \begin{macro}{\@@_lastpagelabel:}
%    \begin{macrocode}
%<*package>
\cs_new_protected:Npn \@@_lastpagelabel:
  {
    \legacy_if:nT { @filesw }
      {
        \exp_args:NNne \exp_args:NNe\iow_now:Nn \@auxout
           {
             \token_to_str:N \new@label@record
               {@tag@LastPage}
               {
                 {abspage} { \int_use:N \g_shipout_readonly_int}
                 {tagmcabs}{ \int_use:N \c@g_@@_MCID_abs_int }
                 {tagstruct}{\int_use:N \c@g_@@_struct_abs_int }
               }
           }
      }
  }  

 \AddToHook{enddocument/afterlastpage}
  {\@@_lastpagelabel:}
%    \end{macrocode}
% \end{macro}
%
% \section{Variables}
% \begin{variable}
%  {
%    \l_@@_tmpa_tl,
%    \l_@@_tmpb_tl,
%    \l_@@_get_tmpc_tl,    
%    \l_@@_get_parent_tmpa_tl
%    \l_@@_get_parent_tmpb_tl    
%    \l_@@_tmpa_str,
%    \l_@@_tmpa_prop,
%    \l_@@_tmpa_seq,
%    \l_@@_tmpb_seq,
%    \l_@@_tmpa_clist,
%    \l_@@_tmpa_int,
%    \l_@@_tmpa_box,
%    \l_@@_tmpb_box
%  }
% A few temporary variables
%    \begin{macrocode}
\tl_new:N    \l_@@_tmpa_tl
\tl_new:N    \l_@@_tmpb_tl
\tl_new:N    \l_@@_get_tmpc_tl
\tl_new:N    \l_@@_get_parent_tmpa_tl
\tl_new:N    \l_@@_get_parent_tmpb_tl
\str_new:N   \l_@@_tmpa_str
\prop_new:N  \l_@@_tmpa_prop
\seq_new:N   \l_@@_tmpa_seq
\seq_new:N   \l_@@_tmpb_seq
\clist_new:N \l_@@_tmpa_clist
\int_new:N   \l_@@_tmpa_int
\box_new:N   \l_@@_tmpa_box
\box_new:N   \l_@@_tmpb_box
%    \end{macrocode}
% \end{variable}
% Attribute lists for the label command. We have a list for
% mc-related labels, and one for structures.
% \begin{macro}{\c_@@_property_mc_clist,\c_@@_property_struct_clist}
%    \begin{macrocode}
\clist_const:Nn \c_@@_property_mc_clist     {tagabspage,tagmcabs,tagmcid}
\clist_const:Nn \c_@@_property_struct_clist {tagstruct,tagstructobj}
%    \end{macrocode}
% \end{macro}
%
% \begin{variable}{\l_@@_loglevel_int}
% This integer hold the log-level and so allows to
% control the messages.
% TODO: a list which log-level shows what is needed. The current behaviour
% is quite ad-hoc.
%    \begin{macrocode}
\int_new:N  \l_@@_loglevel_int
%    \end{macrocode}
% \end{variable}
%
% \begin{variable}
%  {
%    \g_@@_active_space_bool,
%    \g_@@_active_mc_bool,
%    \g_@@_active_tree_bool,
%    \g_@@_active_struct_bool,
%    \g_@@_active_struct_dest_bool
%  }
% These booleans should help to control the global behaviour of tagpdf.
% Ideally it should more or less do nothing if all are false.
% The space-boolean controls the interword space code,
% the mc-boolean activates \cs{tag_mc_begin:n},
% the tree-boolean activates writing the finish code and the pdfmanagement related
% commands, the struct-boolean activates the storing of the structure data.
% In a normal document all should be active, the split is only there for
% debugging purpose.
% Structure destination will be activated automatically if pdf version 2.0 is detected,
% but with the boolean struct-dest-boolean one can suppress them.
% Also we assume currently that they are set only at begin document. But
% if some control passing over groups are needed they could be perhaps
% used in a document too.
% TODO: check if they are used everywhere as needed and as wanted.
%    \begin{macrocode}
\bool_new:N \g_@@_active_space_bool
\bool_new:N \g_@@_active_mc_bool
\bool_new:N \g_@@_active_tree_bool
\bool_new:N \g_@@_active_struct_bool
\bool_new:N \g_@@_active_struct_dest_bool
\bool_gset_true:N \g_@@_active_struct_dest_bool
%    \end{macrocode}
% \end{variable}

% \begin{variable}
%  {
%    \l_@@_active_mc_bool,
%    \l_@@_active_struct_bool,
%    \l_@@_active_socket_bool
%  }
% These booleans should help to control the \emph{local} behaviour of tagpdf.
% In some cases it could e.g. be necessary to stop tagging completely.
% As local booleans they respect groups.
% TODO: check if they are used everywhere as needed and as wanted.
%    \begin{macrocode}
\bool_new:N \l_@@_active_mc_bool
\bool_set_true:N \l_@@_active_mc_bool
\bool_new:N \l_@@_active_struct_bool
\bool_set_true:N \l_@@_active_struct_bool
\bool_new:N \l_@@_active_socket_bool
%    \end{macrocode}
% \end{variable}
%
% \begin{variable}{\g_@@_tagunmarked_bool}
% This boolean controls if the code should try to automatically
% tag parts not in mc-chunk. It is currently only used in luamode.
% It would be possible to used it in generic mode, but this would create
% quite a lot empty artifact mc-chunks.
%    \begin{macrocode}
\bool_new:N \g_@@_tagunmarked_bool
%    \end{macrocode}
% \end{variable}
%
% \section{Variants of l3 commands}
%     \begin{macrocode}
\prg_generate_conditional_variant:Nnn \pdf_object_if_exist:n {e}{T,F,TF}
\cs_generate_variant:Nn \pdf_object_ref:n {e}
\cs_generate_variant:Nn \pdfannot_dict_put:nnn {nne}
\cs_generate_variant:Nn \pdffile_embed_stream:nnn {nee,oee}
\cs_generate_variant:Nn \prop_gput:Nnn {Nee,Nen}
\cs_generate_variant:Nn \prop_put:Nnn  {Nee}
\cs_generate_variant:Nn \prop_item:Nn {No,Ne}
\cs_generate_variant:Nn \seq_set_split:Nnn{Nne}
\cs_generate_variant:Nn \str_set_convert:Nnnn {Nonn, Noon, Nnon }
\cs_generate_variant:Nn \clist_map_inline:nn {on}
%    \end{macrocode}
%
% \section{Label and Reference commands}
% To ease transition to properties we setup internal definition. They can be replaced
% by the property definitions once that is released.
% \begin{macro}{\@@_property_new:nnnn,\@@_property_gset:nnnn,\@@_property_ref:nnn}
% At first a command to define new properties
%    \begin{macrocode}
    \cs_new_eq:NN \@@_property_new:nnnn \property_new:nnnn
%    \end{macrocode}
% For the non-shipout code we need also the option to reset property
%    \begin{macrocode}
    \cs_new_eq:NN \@@_property_gset:nnnn \property_gset:nnnn
%    \end{macrocode}
% The command to reference while giving a local default.
%    \begin{macrocode}
    \cs_new_eq:NN \@@_property_ref:nnn \property_ref:nnn 
    \cs_new_eq:NN \@@_property_ref:nn \property_ref:nn 
%    \end{macrocode}
% The command to record     
%    \begin{macrocode}
    \cs_new_protected:Npn \@@_property_record:nn #1#2
      {
        \@bsphack
        \property_record:nn{#1}{#2}
        \@esphack
      }  
  
%    \end{macrocode}
% And a few variants
%    \begin{macrocode}
\cs_generate_variant:Nn \@@_property_ref:nnn {enn}
\cs_generate_variant:Nn \@@_property_ref:nn {en}
\cs_generate_variant:Nn \@@_property_record:nn {en,eV}
%    \end{macrocode}
% \end{macro}
% 
% \begin{macro}{\@@_property_ref_lastpage:nn}
% A command to retrieve the lastpage label, this will be adapted when there
% is a proper, kernel lastpage label.
%    \begin{macrocode}
\cs_new:Npn \@@_property_ref_lastpage:nn #1 #2
  {
    \@@_property_ref:nnn {@tag@LastPage}{#1}{#2}
  }
%    \end{macrocode}
%\end{macro}

% \section{Setup label attributes}
% \begin{macro}{tagstruct,tagstructobj,tagabspage,tagmcabs,tagmcid}
% This are attributes used by the label/ref system.
% With structures we store the structure number |tagstruct|
% and the object reference |tagstructobj|.
% The second is needed to be able to reference a structure which hasn't been
% created yet. The alternative would be to create the object in such cases,
% but then we would have to check the object existence all the time.
%
% With mc-chunks we store the absolute page number |tagabspage|,
% the absolute id |tagmcabc|, and the id on the page |tagmcid|.
%    \begin{macrocode}
\@@_property_new:nnnn 
  { tagstruct } { now } 
  {0} { \int_use:N \c@g_@@_struct_abs_int }
\@@_property_new:nnnn  { tagstructobj } { now }  {}
  {
    \pdf_object_if_exist:eT {@@/struct/\int_use:N \c@g__tag_struct_abs_int}
      {
        \pdf_object_ref:e{@@/struct/\int_use:N \c@g_@@_struct_abs_int}
      }
  }
\@@_property_new:nnnn  
  { tagabspage } { shipout } 
  {0} { \int_use:N \g_shipout_readonly_int }
\@@_property_new:nnnn  { tagmcabs } { now }
  {0} { \int_use:N \c@g_@@_MCID_abs_int }

\flag_new:n { @@/mcid }     
\@@_property_new:nnnn  {tagmcid } { shipout }
   {0} { \flag_height:n { @@/mcid } }
  
%    \end{macrocode}
% \end{macro}
%

%\section{Commands to fill seq and prop}
% With most engines these are simply copies of the expl3 commands,
% but luatex will overwrite them, to store the data also in lua tables.
% \begin{macro}
%  {
%    \@@_prop_new:N
%    ,\@@_seq_new:N
%    ,\@@_prop_gput:Nnn
%    ,\@@_seq_gput_right:Nn
%    ,\@@_seq_item:cn
%    ,\@@_prop_item:cn
%    ,\@@_seq_show:N
%    ,\@@_prop_show:N
%  }
%    \begin{macrocode}
\cs_set_eq:NN \@@_prop_new:N        \prop_new:N
\cs_set_eq:NN \@@_seq_new:N         \seq_new:N
\cs_set_eq:NN \@@_prop_gput:Nnn     \prop_gput:Nnn
\cs_set_eq:NN \@@_seq_gput_right:Nn \seq_gput_right:Nn
\cs_set_eq:NN \@@_seq_item:cn       \seq_item:cn
\cs_set_eq:NN \@@_prop_item:cn      \prop_item:cn
\cs_set_eq:NN \@@_seq_show:N        \seq_show:N
\cs_set_eq:NN \@@_prop_show:N       \prop_show:N
% cnx temporary needed for latex-lab-graphic code
\cs_generate_variant:Nn \@@_prop_gput:Nnn      { Nen , Nee, Nne , cnn, cen, cne, cno, cnx}
\cs_generate_variant:Nn \@@_seq_gput_right:Nn  { Ne  , No, cn, ce }
\cs_generate_variant:Nn \@@_prop_new:N   { c }
\cs_generate_variant:Nn \@@_seq_new:N    { c }
\cs_generate_variant:Nn \@@_seq_show:N   { c }
\cs_generate_variant:Nn \@@_prop_show:N  { c }
%</package>
%    \end{macrocode}
% \end{macro}
%
% \section{General tagging commands}
%
% \begin{macro}{
%   \tag_stop:,\tag_start:,
%   \tag_stop:n,\tag_start:n}
% We need commands to stop tagging in some places.
% They switch local booleans and also stop the counting of paragraphs.
% The commands keep track of the nesting with a local
% counter. Tagging only is only restarted at the outer level,
% if the current level is 1.
% The commands with argument allow to give a label. This is only used
% in debugging messages to allow to follow the nesting.
% 
% \begin{variable}{\l_@@_tag_stop_int}
% When stop/start pairs are nested we do not want the inner 
% start command to restart tagging. To control this we use a 
% local int: The stop command will increase it. The starting
% will decrease it and only restart tagging, if it is zero.
% This will replace the label version.
%    \begin{macrocode}
%<*package|debug>
%<package>\int_new:N \l_@@_tag_stop_int
%    \end{macrocode}
% \end{variable}
% 
%    \begin{macrocode}
\cs_set_protected:Npn \tag_stop:
  {
%<debug>     \msg_note:nnx {tag / debug }{tag-stop}{ \int_use:N \l_@@_tag_stop_int }      
    \int_incr:N \l_@@_tag_stop_int
    \bool_set_false:N \l_@@_active_struct_bool
    \bool_set_false:N \l_@@_active_mc_bool
    \bool_set_false:N \l_@@_active_socket_bool
    \@@_stop_para_ints:
  }
\cs_set_protected:Npn \tag_start:
  {
    \int_if_zero:nF { \l_@@_tag_stop_int } { \int_decr:N \l_@@_tag_stop_int }
    \int_if_zero:nT { \l_@@_tag_stop_int }
      {
        \bool_set_true:N \l_@@_active_struct_bool
        \bool_set_true:N \l_@@_active_mc_bool
        \bool_set_true:N \l_@@_active_socket_bool
        \@@_start_para_ints:
      }
%<debug>    \msg_note:nnx {tag / debug }{tag-start}{ \int_use:N \l_@@_tag_stop_int }                   
  }
\cs_set_eq:NN\tagstop\tag_stop:
\cs_set_eq:NN\tagstart\tag_start:  
%    \end{macrocode}
%    \begin{macrocode}
\cs_set_protected:Npn \tag_stop:n #1
  {
%<debug>    \msg_note:nnxx {tag / debug }{tag-stop}{ \int_use:N \l_@@_tag_stop_int }{#1}           
    \int_incr:N \l_@@_tag_stop_int
    \bool_set_false:N \l_@@_active_struct_bool
    \bool_set_false:N \l_@@_active_mc_bool
    \bool_set_false:N \l_@@_active_socket_bool
    \@@_stop_para_ints:
  }
\cs_set_protected:Npn \tag_start:n #1
  {
    \int_if_zero:nF { \l_@@_tag_stop_int } { \int_decr:N \l_@@_tag_stop_int }
    \int_if_zero:nT { \l_@@_tag_stop_int }
      {
        \bool_set_true:N \l_@@_active_struct_bool
        \bool_set_true:N \l_@@_active_mc_bool
        \bool_set_true:N \l_@@_active_socket_bool
        \@@_start_para_ints:
      } 
%<debug>    \msg_note:nnxx {tag / debug }{tag-start}{ \int_use:N \l_@@_tag_stop_int }{#1}             
  }
%</package|debug>
%<*base>
\cs_new_protected:Npn \tag_stop:{}
\cs_new_protected:Npn \tag_start:{}
\cs_new_protected:Npn \tagstop{}
\cs_new_protected:Npn \tagstart{}
\cs_new_protected:Npn \tag_stop:n  #1 {}
\cs_new_protected:Npn \tag_start:n #1 {}
%</base>
%    \end{macrocode}
% \end{macro}
%
%
% \section{Keys for tagpdfsetup}

% TODO: the log-levels must be sorted
%
% \begin{macro}
%  {
%    activate-space (setup-key),
%    activate-mc (setup-key),
%    activate-tree (setup-key),
%    activate-struct (setup-key),
%    activate-all (setup-key),
%    no-struct-dest (setup-key),
%  }
%  Keys to (globally) activate tagging.
%  |activate-space| activates the additional parsing needed for
%  interword spaces. It is not documented, the parsing is currently
%  implicitly activated by the known key |interwordspace|, as the code will
%  perhaps move to some other place, now that it is better separated.
%  |no-struct-dest| allows to suppress structure destinations.
%    \begin{macrocode}
%<*package>
\keys_define:nn { @@ / setup }
  {
    activate-space  .bool_gset:N = \g_@@_active_space_bool,
    activate-mc     .bool_gset:N = \g_@@_active_mc_bool,
    activate-tree   .bool_gset:N = \g_@@_active_tree_bool,
    activate-struct .bool_gset:N = \g_@@_active_struct_bool,
    activate-all    .meta:n =
      {activate-mc={#1},activate-tree={#1},activate-struct={#1}},
    activate-all  .default:n = true,
    no-struct-dest .bool_gset_inverse:N = \g_@@_active_struct_dest_bool,
 %    \end{macrocode}
% \end{macro}
% \begin{macro}{log (setup-key)}
% The |log| takes currently the values |none|, |v|, |vv|, |vvv|, |all|.
% The description of the log levels is in tagpdf-checks.
%    \begin{macrocode}
    log             .choice:,
    log / none      .code:n = {\int_set:Nn \l_@@_loglevel_int { 0 }},
    log / v         .code:n =
      {
        \int_set:Nn \l_@@_loglevel_int { 1 }
        \cs_set_protected:Nn \@@_check_typeout_v:n { \iow_term:e {##1} }
      },
    log / vv        .code:n = {\int_set:Nn \l_@@_loglevel_int { 2 }},
    log / vvv       .code:n = {\int_set:Nn \l_@@_loglevel_int { 3 }},
    log / all       .code:n = {\int_set:Nn \l_@@_loglevel_int { 10 }},
%    \end{macrocode}
% \end{macro}
% \begin{macro}{tagunmarked (setup-key)}
% This key allows to set if (in luamode) unmarked text should be
% marked up as artifact. The initial value is true.
%    \begin{macrocode}
    tagunmarked     .bool_gset:N = \g_@@_tagunmarked_bool,
    tagunmarked     .initial:n  = true,
%    \end{macrocode}
% \end{macro}
% \begin{macro}{tabsorder (setup-key)}
% This sets the tabsorder on a page. The values are |row|, |column|, |structure| (default)
% or |none|. Currently this is set more or less globally. More finer control can be
% added if needed.
%    \begin{macrocode}
    tabsorder       .choice:,
    tabsorder / row       .code:n =
      \pdfmanagement_add:nnn { Page } {Tabs}{/R},
    tabsorder / column    .code:n =
      \pdfmanagement_add:nnn { Page } {Tabs}{/C},
    tabsorder / structure .code:n =
      \pdfmanagement_add:nnn { Page } {Tabs}{/S},
    tabsorder / none      .code:n =
      \pdfmanagement_remove:nn {Page} {Tabs},
    tabsorder       .initial:n = structure,
    uncompress      .code:n = { \pdf_uncompress:  },
  }
%    \end{macrocode}
% \end{macro}
% \section{loading of engine/more dependent code}
%    \begin{macrocode}
\sys_if_engine_luatex:T
  {
    \file_input:n {tagpdf-luatex.def}
  }
%</package>
%    \end{macrocode}
%    \begin{macrocode}
%<*mcloading>
\bool_if:NTF \g_@@_mode_lua_bool
  {
   \RequirePackage {tagpdf-mc-code-lua}
  }
  {
   \RequirePackage {tagpdf-mc-code-generic} %
  }
%</mcloading>
%<*debug>
\bool_if:NTF \g_@@_mode_lua_bool
  {
   \RequirePackage {tagpdf-debug-lua}
  }
  {
   \RequirePackage {tagpdf-debug-generic} %
  }
%</debug>
%    \end{macrocode}
% \end{implementation}
% \PrintIndex
